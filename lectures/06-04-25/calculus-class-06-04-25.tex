\documentclass[11pt,oneside]{article}
\usepackage{amsmath,amssymb,amsthm,geometry,enumitem,hyperref,mathtools,framed}
\geometry{margin=1in}
\usepackage{lmodern}
\usepackage{microtype}

\newtheorem{theorem}{Theorem}[section]
\newtheorem{lemma}[theorem]{Lemma}
\newtheorem{corollary}[theorem]{Corollary}
\newtheorem{proposition}[theorem]{Proposition}
\theoremstyle{definition}
\newtheorem{definition}[theorem]{Definition}
\newtheorem{example}[theorem]{Example}
\newtheorem{remark}[theorem]{Remark}
\theoremstyle{remark}
\newtheorem*{administrative_note}{Administrative Announcement}

\setlength{\parskip}{0.7em}
\setlength{\parindent}{2em}

\begin{document}

\begin{center}
    {\LARGE \textbf{Improper Integrals: Definitions, Techniques, and Convergence}}

    \vspace{0.5em}

    \textbf{Lecture Notes and Announcements}

    \vspace{0.3em}
    \emph{Prepared for Undergraduate Calculus Students}
\end{center}

\bigskip

\begin{administrative_note}
\vspace{0.8em}
\begin{itemize}[leftmargin=2em]
    \item \textbf{Course Update:} Until further notice, classes will be held via Zoom and not in person. Please check your course email for the meeting link and updates.
    \item \textbf{Office Hours:} If you have further questions or would like to discuss additional problems, you are encouraged to reach out after class or during designated office hours.
    \item \textbf{Exercise:} Some results and techniques (such as convergence for certain $\alpha$ in improper integrals) are left for you to prove at home with the guidance below.
    \item \textbf{General Note:} If you have any concerns or questions about course structure, scheduling, logistics, or assessment, please communicate with the instructional staff as early as possible.
\end{itemize}
\vspace{0.7em}
\end{administrative_note}

\bigskip

\section{Motivation: Why Improper Integrals?}

In your calculus journey so far, you have learned to compute definite integrals of continuous functions over closed, bounded intervals.
But what happens when:
\begin{itemize}
    \item The interval of integration is infinite, e.g., $\int_1^{\infty} \frac{1}{x^2}\,dx$?
    \item The function is unbounded at a point within or at the endpoint of the interval, e.g., $\int_0^1 \frac{1}{\sqrt{x}}\,dx$?
\end{itemize}

Such integrals occur frequently in applications and in theoretical mathematics: for instance, in probability theory (normal distribution), in analyzing series (by comparison with integrals), and in physics (potential fields, etc.). To handle these, we \textbf{extend the definition of the definite integral} using limits, entering the world of \emph{improper integrals}.

\section{Improper Integrals: Definitions and First Examples}

\subsection{Improper Integrals at Finite Endpoints (Unbounded Functions)}

Suppose $f$ is defined on an interval $[a, b)$ (that is, it may not be defined or may blow up at $b$) and $f$ is \emph{bounded} and \emph{integrable} on every subinterval $[a, t]$ for $t < b$.
We define the \emph{improper integral}
\[
\int_a^b f(x)\,dx := \lim_{t \to b^-} \int_a^t f(x)\,dx
\]
provided this limit exists (and is finite!).

Likewise, if $f$ is problematic (e.g., becomes infinite) at $a$, one writes:
\[
\int_a^b f(x)\,dx := \lim_{t \to a^+} \int_t^b f(x)\,dx
\]
Again, provided the limit exists.

\paragraph{Intuitive picture:}
We avoid values where $f$ is not defined by integrating short of the problematic endpoint, and take the limit as we approach it.

\vspace{0.7em}
\noindent
\textbf{Fundamental Question:} Does the limit exist (and is it finite)? If yes, we say the improper integral \textbf{converges}. If the limit does not exist or is infinite, we say the improper integral \textbf{diverges}.

\subsection{A Classic Example: $\displaystyle \int_0^1 \frac{1}{\sqrt{x}}\,dx$}

\begin{example}
Let us consider the function $f(x) = \dfrac{1}{\sqrt{x}}$ on the interval $(0,1]$. Note that $f$ is \href{https://en.wikipedia.org/wiki/Unbounded_function}{\emph{unbounded}} as $x \to 0^+$: $f(x) \to \infty$.

We define the improper integral:
\[
\int_0^1 \frac{1}{\sqrt{x}}\,dx := \lim_{t\to 0^+} \int_t^1 \frac{1}{\sqrt{x}}\,dx
\]
We compute:
\begin{align*}
\int_t^1 \frac{1}{\sqrt{x}}\,dx
    &= \left[2\sqrt{x}\right]_{x=t}^{x=1} = 2\sqrt{1} - 2\sqrt{t} = 2 - 2\sqrt{t} \\
\Rightarrow \int_0^1 \frac{1}{\sqrt{x}}\,dx
    &= \lim_{t\to 0^+} (2 - 2\sqrt{t}) = 2
\end{align*}

\textbf{Conclusion:} Though $f(x)$ blows up at $0$, the area under the curve from $0$ to $1$ is \emph{finite}: the improper integral converges to $2$.

\textbf{Visual intuition:} The graph of $1/\sqrt{x}$ on $(0,1]$ rises steeply near $x=0$, but the ``area'' does not become infinite.
\end{example}

\begin{remark}
Why can we compute the area, even though $f$ is not defined at $x=0$? The improper integral is defined as a limit that never actually plugs in $x=0$. Instead, it considers areas starting at $x=t>0$, then lets $t \to 0^+$.
\end{remark}

\section{Improper Integrals: General Definition}

\begin{definition}[Improper Integral of the First Kind]
Let $f:[a, b) \to \mathbb{R}$ be a function bounded and integrable on $[t, b]$ for all $t > a$. Then
\[
\int_a^b f(x)\,dx := \lim_{t \to a^+} \int_t^b f(x)\,dx
\]
if this limit \emph{exists} and is finite.

If not, we say the integral \emph{diverges}.
\end{definition}

A similar definition is given for $f$ defined on $(a, b]$, with a possible singularity at $b$:
\[
\int_a^b f(x)\,dx := \lim_{t \to b^-} \int_a^t f(x)\,dx
\]

\begin{remark}
An interval $[a, b)$ is called a \emph{half-open interval}. At the ``open'' end, the function may not be defined or may be unbounded. Our limits always approach from within the domain of $f$.
\end{remark}

\begin{example}[Singularity at the Right Endpoint]
Calculate
\[
\int_0^1 \frac{1}{\sqrt{1-x^2}}\,dx
\]

\textbf{Solution:}
Here, the integrand $f(x) = \frac{1}{\sqrt{1-x^2}}$ is unbounded at $x=1$.

We set
\[
\int_0^1 \frac{1}{\sqrt{1-x^2}}\,dx = \lim_{t \to 1^-} \int_0^t \frac{1}{\sqrt{1-x^2}}\,dx
\]

The antiderivative is $\arcsin x$, so:
\begin{align*}
\int_0^t \frac{1}{\sqrt{1-x^2}}\,dx = \arcsin t - \arcsin 0 = \arcsin t
\end{align*}
Therefore,
\[
\int_0^1 \frac{1}{\sqrt{1-x^2}}\,dx = \lim_{t \to 1^-} \arcsin t = \arcsin 1 = \frac{\pi}{2}
\]

\textbf{Conclusion:} The improper integral converges to $\frac{\pi}{2}$. This is, in fact, the area of a quarter of the unit circle.
\end{example}

\begin{example}[Divergence at Endpoint]
Consider
\[
\int_0^1 \frac{1}{x^2}\,dx
\]
We have (for $t > 0$):
\[
\int_t^1 \frac{1}{x^2}\,dx = \left[ -\frac{1}{x} \right]_{t}^1 = -1 + \frac{1}{t}
\]
As $t \to 0^+$, $1/t \to \infty$, so:
\[
\int_0^1 \frac{1}{x^2}\,dx = \infty
\]
\textbf{Conclusion:} The improper integral \emph{diverges}; the area under $1/x^2$ near $0$ is infinite.
\end{example}

\section{A Family of Examples: Powers of $x$ at the Endpoint}\label{sec:power_examples}

A natural question is: For which $p > 0$ does the integral
\[
\int_0^1 \frac{1}{x^p}\,dx
\]
converge?

Let's analyze this in full generality:

\begin{theorem}
Let $\alpha \in \mathbb{R}$. Consider the improper integral
\[
I(\alpha) = \int_0^b x^{-\alpha}\,dx
\]
for $b > 0$. Then:
\begin{itemize}
    \item[(a)] If $\alpha < 1$, the integral converges.
    \item[(b)] If $\alpha \geq 1$, the integral diverges.
\end{itemize}
\end{theorem}

\begin{proof}
We compute, for $t \in (0, b)$:
\[
I(\alpha, t) = \int_t^b x^{-\alpha}\,dx =
\begin{cases}
    \left. \frac{x^{1-\alpha}}{1-\alpha} \right|_{x=t}^{x=b}, & \text{if } \alpha \neq 1 \\
    \left. \ln x \right|_{x=t}^{x=b}, & \text{if } \alpha = 1
\end{cases}
\]

For $\alpha \neq 1$:
\[
I(\alpha, t) = \frac{b^{1-\alpha} - t^{1-\alpha}}{1-\alpha}
\]
Take the limit as $t \to 0^+$. Suppose $\alpha < 1$ so $1-\alpha>0$:
\[
t^{1-\alpha} \to 0, \quad I(\alpha) = \frac{b^{1-\alpha}}{1-\alpha}
\]
\emph{Converges.}

But if $\alpha > 1$, then $1-\alpha<0$, so $t^{1-\alpha} \to \infty$, and the integral diverges to infinity.

In the special case $\alpha=1$:
\[
I(1, t) = \ln b - \ln t = \ln\left(\frac{b}{t}\right) \to \infty \text{ as } t \to 0^+
\]
Again, the integral diverges.

\end{proof}

\begin{example}
Consider:
\begin{enumerate}
    \item For $\alpha = \frac{1}{2}$, $\int_0^1 \dfrac{1}{x^{1/2}}\,dx$ converges (see earlier Example).
    \item For $\alpha = 2$, $\int_0^1 \dfrac{1}{x^2}\,dx$ diverges (computed above).
    \item For $\alpha = 1$, $\int_0^1 \dfrac{1}{x}\,dx$ diverges (see previous discussion).
\end{enumerate}
\end{example}

\begin{remark}
\textbf{Intuition:} The function rises \emph{too fast} near $x=0$ when the exponent is too large ($\alpha \geq 1$), so the area accumulates without bound.
\end{remark}

\section{Root and Trigonometric Integrals: Further Illustrations}

\begin{example}[Improper Integral involving Trigonometric Functions]
Compute:
\[
\int_0^{\pi/2} \frac{\cos x}{\sqrt{1 - \sin x}}\,dx
\]

\textit{Solution:}
Let $u = 1 - \sin x$, so $du = -\cos x\,dx$, or $dx = -\frac{du}{\cos x}$.

Alternatively, set $y = \sin x$, so $dy = \cos x\,dx$. When $x = 0$, $y = 0$. When $x = \pi/2$, $y = 1$.

Thus,
\[
\int_0^{\pi/2} \frac{\cos x}{\sqrt{1 - \sin x}}\,dx = \int_{y=0}^{y=1} \frac{dy}{\sqrt{1 - y}} 
\]
This is the same as $\int_0^1 (1-y)^{-1/2}\,dy$, which we've already calculated:

\[
\int_0^1 (1-y)^{-1/2}\,dy = \lim_{t\to 1^-} \int_0^t (1-y)^{-1/2}\,dy = \lim_{t\to 1^-} \left[-2\sqrt{1-y}\right]_0^t = 2 - 0 = 2
\]

\textbf{Conclusion:} The value is $2$.
\end{example}

\section{Advanced Example: Substitution and Complex Integrands}

\begin{example}
Evaluate:
\[
\int_{\ln 2}^{\ln 9} \frac{1}{\sqrt[3]{e^x-1}}\,dx
\]

\textit{Solution:} Let $t = e^x$, so $dt = e^x dx$ and $dx = dt/t$. When $x = \ln 2$, $t = 2$; when $x = \ln 9$, $t = 9$. Thus,
\[
\int_{\ln 2}^{\ln 9} \frac{1}{\sqrt[3]{e^x-1}}\,dx
 = \int_{2}^{9} \frac{1}{\sqrt[3]{t-1}} \cdot \frac{dt}{t}
\]

This is still not conveniently integrable, but represents a more tractable substitution. Further substitutions (for instance, $u = (t-1)^{1/3}$) may or may not lead to an elementary expression. The essential lesson is the flexible use of substitution in analyzing improper integrals.

\end{example}

\section{Comparison Tests for Improper Integrals}

Often, one cannot compute an improper integral directly, but wishes to know whether it converges. The following is a key tool:

\begin{theorem}[Comparison Test for Improper Integrals]
Let $f$ and $g$ be non-negative functions on $[a, b)$. Suppose $f(x) \leq g(x)$ for all $x$ in some right-hand neighborhood $[a, a+\delta)$ for some $\delta > 0$, and that both are integrable on $[t, b]$ for all $t \in (a,b)$. Then:
\begin{enumerate}
    \item If $\int_a^b g(x)\,dx$ converges, then so does $\int_a^b f(x)\,dx$.
    \item If $\int_a^b f(x)\,dx$ diverges, then so does $\int_a^b g(x)\,dx$.
\end{enumerate}
\end{theorem}

\begin{remark}
In many practical cases, the inequality holds on the whole interval.
\end{remark}

\subsection{Worked Examples}

\begin{example}
Determine the convergence of
\[
\int_1^2 \frac{1}{\ln x}\,dx
\]
\textbf{Solution:} As $x \to 1^+$, $\ln x \sim x-1$. Therefore, for $x$ near $1$,
\[
\frac{1}{\ln x} > \frac{1}{x-1}
\]
We know that
\[
\int_1^2 \frac{1}{x-1}\,dx = \left[ \ln(x-1) \right]_{x=1}^{x=2} = \ln 1 - \ln 0 = -\infty
\]
The improper integral diverges, so by the comparison test, so does $\int_1^2 \frac{1}{\ln x}\,dx$.

\emph{Moral:} Even though $1/\ln x$ grows slowly, it's still too wild near $x=1$ for the area to be finite!
\end{example}

\begin{example}
Analyze the convergence of:
\[
\int_0^1 \frac{1}{\sqrt{x} + 4x^{1/5} + 2x^{1/3}}\,dx
\]

Note that as $x \to 0^+$, $\sqrt{x} \ll x^{1/5} \ll x^{1/3}$, so the denominator is approximately $\sqrt{x}$, which is smallest.

Thus,
\[
\frac{1}{\sqrt{x} + 4x^{1/5} + 2x^{1/3}} < \frac{1}{\sqrt{x}}
\]

From previous calculations, $\int_0^1 1/\sqrt{x}\,dx$ converges. So by the comparison test, our integral converges as well.
\end{example}

\begin{example}
Evaluate the convergence of
\[
\int_0^1 \frac{x^2}{\sqrt{x^2 + 13}}\,dx
\]
For $x \to 0$, the numerator $x^2 \to 0$ much faster than the denominator $\sqrt{x^2 + 13} \to \sqrt{13}$. Thus the function is actually continuous at $x=0$ and the integral is proper (finite). Hence, the integral converges.
\end{example}

\subsection{Theoretical Insights}

\begin{remark}
\textbf{Why care about convergence?} Sometimes we do not need the actual value of an integral---only to know whether it is finite! For example, in probability, some quantities are only meaningful if the area under a density curve is finite.
\end{remark}

\section{Applications and Connections}

\subsection{Improper Integrals in Probability}

Consider the normal distribution (bell curve), whose density function is 
\[
f(x) = \frac{1}{\sqrt{2\pi}\sigma} e^{-\frac{(x-\mu)^2}{2\sigma^2}}
\]
Defined for all $x \in \mathbb{R}$. To find the probability that a random variable $X$ lies in $[a,b]$:
\[
P(a \leq X \leq b) = \int_a^b f(x)\,dx
\]
Since $f$ is defined on all of $\mathbb{R}$, the total area:
\[
\int_{-\infty}^\infty f(x)\,dx = 1
\]
is an improper integral! Convergence of this area is crucial for $f$ to serve as a probability density.

\subsection{Improper Integrals in Statistics}

In statistics, many distributions (e.g., exponential, normal, Cauchy) require integrating over an infinite or semi-infinite range to compute probabilities or expected values. Improper integrals are essential in formalizing such computations.

\section{Concluding Remarks}

We have seen that improper integrals elegantly generalize the concept of area and total accumulation to include scenarios where the integrand or integration limits exhibit ``improper'' behavior. Mastery of these ideas will serve you in numerous branches of analysis, probability, and applied mathematics.

If you have questions about course material or logistics, do not hesitate to reach out via email or in virtual office hours.

\medskip

\hrule
\vspace{0.5em}
\noindent
\textbf{Administrative Announcement (Summary):}
\begin{itemize}
    \item Next class will be held on \textbf{Zoom}.
    \item Additional assignment guidelines and results to be posted online.
    \item If you are unsure about the convergence of a specific improper integral, use comparison tests or ask during office hours.
\end{itemize}
\vspace{0.5em}
\hrule

\end{document}