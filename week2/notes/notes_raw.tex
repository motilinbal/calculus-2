% This LaTeX document needs to be compiled with XeLaTeX.
\documentclass[10pt]{article}
\usepackage[utf8]{inputenc}
\usepackage{ucharclasses}
\usepackage{amsmath}
\usepackage{amsfonts}
\usepackage{amssymb}
\usepackage[version=4]{mhchem}
\usepackage{stmaryrd}
\usepackage{graphicx}
\usepackage[export]{adjustbox}
\graphicspath{ {./images/} }
\usepackage[fallback]{xeCJK}
\usepackage{polyglossia}
\usepackage{fontspec}
\IfFontExistsTF{Noto Serif CJK TC}
{\setCJKmainfont{Noto Serif CJK TC}}
{\IfFontExistsTF{STSong}
  {\setCJKmainfont{STSong}}
  {\IfFontExistsTF{Droid Sans Fallback}
    {\setCJKmainfont{Droid Sans Fallback}}
    {\setCJKmainfont{SimSun}}
}}

\setmainlanguage{norwegian}
\setotherlanguages{dutch, english, hebrew}
\IfFontExistsTF{Noto Serif Hebrew}
{\newfontfamily\hebrewfont{Noto Serif Hebrew}}
{\IfFontExistsTF{Arial Hebrew}
  {\newfontfamily\hebrewfont{Arial Hebrew}}
  {\IfFontExistsTF{FreeSerif}
    {\newfontfamily\hebrewfont{FreeSerif}}
    {\newfontfamily\hebrewfont{David}}
}}
\IfFontExistsTF{CMU Serif}
{\newfontfamily\lgcfont{CMU Serif}}
{\IfFontExistsTF{DejaVu Sans}
  {\newfontfamily\lgcfont{DejaVu Sans}}
  {\newfontfamily\lgcfont{Georgia}}
}
\setDefaultTransitions{\lgcfont}{}
\setTransitionsFor{Hebrew}{\hebrewfont}{\lgcfont}

\begin{document}
$$
\begin{array}{cc}
24 / 3 / 25 & \text { poj산 } \\
11-12 & \text { pe } \\
4416 &
\end{array}
$$

$: 178 x^{\circ} 18100$\\
\includegraphics[max width=\textwidth, center]{2025_04_04_18c0ad40a2047306fd65g-1(2)}

$$
\begin{aligned}
& F(x)=\int f(x) d x
\end{aligned}
$$

\begin{itemize}
  \item $\int x \cdot d x==\frac{x^{2}}{2}$\\
: תlland 13
  \item $\int x \cdot d x=\frac{x^{2}}{2}+6 \rightarrow\left(\frac{x^{2}}{2}+6\right)^{\prime}=x$\\
$\int f(x) d x=F(x)+c$\\
\includegraphics[max width=\textwidth, center]{2025_04_04_18c0ad40a2047306fd65g-1(1)}\\
\includegraphics[max width=\textwidth, center]{2025_04_04_18c0ad40a2047306fd65g-1}\\
ble vin'ze\\
gis刀
\end{itemize}

$$
\begin{aligned}
& *\left(\frac{\cos 2 x}{2}\right)^{\prime}=\frac{-\sin (2 x) \cdot 2}{2}=-\sin (2 x) \rightarrow \int(-\sin 2 x) d x=\frac{\cos (2 x)}{2}+c\left\{\begin{array}{l}
\text { y/lc } \\
\text { yolk }
\end{array}\right. \\
& \alpha\left(-\sin ^{2} x\right)^{\prime}=-2 \sin x \cdot \cos x=-\sin 2 x \rightarrow \int(-\sin 2 x) d x=-\sin ^{2} x+c
\end{aligned}
$$

: Krdl3\\
\includegraphics[max width=\textwidth, center]{2025_04_04_18c0ad40a2047306fd65g-1(3)}\\
(1) $\int x^{\alpha} d x=\frac{x^{\alpha}-1}{\alpha+1}-c \quad \int \frac{1}{x} d x=\ln |x|+c$ :P"3:N PMASGila

$$
\int \frac{1}{g(x)} d x=\int \frac{d x}{g(x)}: \ln \cdot 0
$$

(2) $(\sin x)^{\prime}=\cos x \quad(\cos x)^{\prime}=-\sin x \quad(-\cos x)^{\prime}=\sin x$

$$
\int \sin x d x=-\cos x+c \quad \int \cos x d x=\sin x+c
$$

(3) $\int \frac{d x}{\sqrt{1-x^{2}}}=\arcsin x+c \quad \int \frac{d x}{1+x^{2}}=\arctan x+c$

$$
\begin{aligned}
& \left(c_{1} F(x)+c_{2} G(x)\right)^{\prime}=C_{1} F^{\prime}(x)+c_{2} G^{\prime}(x) \text {-e д13. }
\end{aligned}
$$

\begin{itemize}
  \item $\int(x-3 \sin x) d x=\int x d x-3 \int \sin x d x=\frac{x^{2}}{2}-3(-\cos x)+c=\frac{x^{2}}{2}+3 \cos x+c$ (arc):ca
\end{itemize}

$$
\text { * } \int(\cos 2 x) d x=\frac{\sin 2 x}{2}+c \quad f(x)=\cos x \quad F(x)=\sin x
$$

$$
a=2 \quad b=0
$$

$$
\text { * } \int \frac{d x}{5 x+7}=\frac{1}{5} \ln |5 x+7|+c \quad f(x)=\frac{1}{x} \quad F(x)=\ln |x|
$$

$$
a=5 b=7
$$

$$
\text { * } \int \frac{d x}{\sqrt{4-x^{2}}}=\int \frac{d x}{2 \sqrt{1-\left(\frac{x}{2}\right)^{2}}}=\frac{1}{2} \cdot \int f\left(\frac{x}{2}\right) d x=\frac{\frac{1}{2} \cdot \arcsin \left(\frac{x}{2}\right)+c}{\frac{1}{2}}=\arcsin \left(\frac{x}{2}\right)+c
$$

$$
f(x)=\frac{1}{\sqrt{1-x^{2}}}
$$

$$
F(x)=\arcsin x
$$

$$
\begin{aligned}
& (\arcsin x+c)^{\prime}=\frac{1}{\sqrt{1-\frac{x^{2}}{4}}} \cdot \frac{1}{2}=\frac{1}{2 \sqrt{1-\frac{x^{2}}{4}}}=\frac{1}{\sqrt{4-x^{2}}} \\
& * \int \frac{d x}{9+x^{2}}=\frac{1}{9} \cdot \frac{1}{1+\left(\frac{x}{3}\right)^{2}}=\frac{1}{9} \cdot \int f\left(\frac{x}{3}\right) d x=\frac{1}{9} \frac{F\left(\frac{x}{3}\right)}{\frac{1}{3}}=\frac{1}{3} \arctan \left(\frac{x}{3}\right)-c S
\end{aligned}
$$

:ñse "o plas)

$$
\frac{d x}{1+x^{2}}=\arctan x+c
$$

$$
f(x)=\frac{1}{1+x^{2}}
$$

$$
F(x)=\arctan x
$$

$$
\left(\frac{1}{3} \arctan \left(\frac{x}{3}\right)+c\right)^{\prime}=\frac{1}{3} \cdot \frac{1}{1+\left(\frac{x}{3}\right)^{2}} \cdot \frac{1}{3}=\frac{1}{9} \cdot \frac{1}{1+\left(\frac{x}{3}\right)^{2}}=\frac{1}{9+x^{2}}
$$

$$
\begin{aligned}
& \left(e^{x}\right)^{\prime}=e^{x} \rightarrow \int e^{x} d x=e^{x}+c \\
& \left(a^{x}\right)^{\prime}=\left(e^{\ln a \cdot x}\right)=e^{\ln a \cdot x} \cdot \ln a=a^{x} \cdot \ln a \rightarrow \int a^{x} d x=\frac{a^{x}}{\ln a}+c \\
& * \int 3^{5 x-2} d x=\frac{3^{5 x-2}}{5 \cdot \ln 3}+c \quad 3^{5 x+2} \\
& 3^{5 x-2} d x=9 \cdot 3^{5 x}=9 \cdot 243^{x} \rightarrow \int 3^{5 x-2} d x^{-} \int 9 \cdot 243^{x} d x=\frac{9 \cdot 243^{x}}{\ln (243)}+c \\
& 5 \cdot \ln 3
\end{aligned}
$$

$$
f(x)=3^{x} \quad a=5 \quad b=2
$$

$$
F(x)=\frac{3^{x}}{\ln 3}
$$

$$
\begin{aligned}
& \text { * } \int \frac{x+1}{x} d x=\int\left(1+\frac{1}{x}\right) d x=\int 1 d x+\int \frac{d x}{x}=x+\ln |x|+c \\
& \text { * } \int \frac{x^{4}}{x^{2}+1} d x=\int \frac{x^{4}-1+1}{x^{2}+1} d x=\int \frac{\left(x^{2}-1\right)\left(x^{2}-1\right)+1}{x^{2}+1} d x=\int\left(x^{2}-1+\frac{1}{x^{2}+1}\right) d x=\int x^{2} d x-\int 1 d x+\int \frac{d x}{x^{2}+1} \\
& =\frac{x^{3}}{3}-x+\arctan (x)+c \\
& \int f(a x+b)=\frac{1}{a} F(a x+b)+c: \text { (nnenen } \iint_{2} N \text { trif) Ro1) }\left(r_{3}\right. \\
& (F(2 x+3))^{\prime}=f(2 x-3) \cdot 2 \quad F^{\prime}(x)=f(x) \rightarrow S f(2 x+3) d x=\frac{F(2 x+3)}{2}+c
\end{aligned}
$$

$$
\text { * } \int\left(x+\cos ^{2} \frac{x}{2}\right) d x=\int\left(x+\frac{1}{2}+\frac{\cos x}{2}\right)=\frac{x^{2}}{2}+\frac{x}{2}+\frac{\sin x}{2}+c
$$

$$
\left\{\begin{array}{r}
\cos 2 \alpha=2 \cos ^{2} \alpha-1 \\
1-2 \sin ^{2} \alpha \\
\cos ^{2} \alpha-\sin ^{2} \alpha \\
\cos ^{2} \alpha=\frac{1-\cos 2 \alpha}{2}
\end{array}\right.
$$

$$
\begin{aligned}
& \text { * } \int\left(x-\cos ^{2} x\right) d x=\cos ^{2} x=\frac{1+\cos 2 x}{2}=\int\left(x+\frac{1}{2}+\frac{\cos (2 x)}{2}\right) d x \\
& =\frac{x^{2}}{2}+\frac{x}{2}+\frac{\sin 2 x}{2 \cdot 2}=\frac{x^{2}+x}{2}+\frac{\sin 2 x}{4}+c
\end{aligned}
$$

\begin{center}
\includegraphics[max width=\textwidth]{2025_04_04_18c0ad40a2047306fd65g-3}
\end{center}

$$
\begin{aligned}
& (F(x) \cdot G(x))^{\prime}=F^{\prime}(x) \cdot G(x)+F(x) \cdot G^{\prime}(x) \\
& \int F^{\prime}(x) G(x) d x=F(x) G(x)-\int F(x) G^{\prime}(x)
\end{aligned}
$$

$$
\int x \cdot e^{x} \cdot d x: 1 \quad|=r| l z
$$

$\int x \cdot e^{x} \cdot d x=\frac{x^{2}}{2} \cdot e^{x}-\int \frac{x^{2}}{2} e^{x}$

$$
x=F^{\prime}(x): 1 \text { p/2 }
$$

inno ik Sok ju) jrike, $\quad \begin{array}{ll} & F(x)=\frac{x^{2}}{2} \\ e^{x}=G\left(x^{2}\right)\end{array}$

$$
\begin{aligned}
& \int x \cdot e^{x} \cdot d x=e^{x} \cdot x-\int e^{x} \cdot 1 d x \\
= & e^{x} \cdot x-e^{x}+c=e^{x}(x-1)+c
\end{aligned}
$$

$$
\begin{aligned}
& e^{x}=F^{\prime}(x) \quad: 2 \quad{ }^{2} 3 \\
& F(x)=e^{x} \\
& x=G(x)
\end{aligned}
$$

$$
\begin{aligned}
& \int_{F^{\prime}(x)} e^{x} \cdot x_{G(x)}^{x^{2}} \cdot d x=e^{x} x^{2}-\int e^{x} \cdot 2 x \cdot d x=e^{x} x^{2}-2 \cdot e^{x}(x-1)+c
\end{aligned}
$$


\end{document}

\documentclass[10pt]{article}
\usepackage[utf8]{inputenc}
\usepackage[T1]{fontenc}
\usepackage{amsmath}
\usepackage{amsfonts}
\usepackage{amssymb}
\usepackage[version=4]{mhchem}
\usepackage{stmaryrd}

\begin{document}
$: 2$ IXID

$$
\int F^{\prime}(x) G(x) d x=F(x) G(x)-\int F(x) G^{\prime}(x) d x \quad: \rho \cdot p / n ? \quad \text { n’ } 3 \neg \imath G \cdot / c
$$

$$
\left(F(G(x))^{\prime}=F^{\prime}(G(x)) G^{\prime}(x) \Rightarrow \int F^{\prime}(G(x)) G^{\prime}(x) d x=F(G(x))=c\right.
$$

$$
t=G(x) \quad \frac{d t}{d x}=t^{\prime}(x)=G^{\prime}(x) \Rightarrow \int F^{\prime}(t) d t=F(t)+c=F(G(x))+c
$$

\begin{itemize}
  \item $\int \sin ^{t^{2} x} \cdot \frac{d t}{\cos x d x}=\int t^{2} d t=\frac{t^{3}}{3}=\frac{\sin ^{3} x}{3}+c \quad\left(\frac{\sin ^{3} x}{3}\right)=\sin ^{2} x \cos x \quad$ : (s) $t=\sin x$
\end{itemize}

$$
\frac{d t}{d x}=\cos x
$$

$$
\text { * } \int^{d x} \frac{e^{x}}{e^{2 x}+1} d x=\int \frac{d t}{t^{2}-1}=\arctan (t)=\arctan \left(e^{x}\right)+c \quad\left(\arctan \left(e^{x}\right)\right)^{\prime}=
$$

$$
t=e^{x} \quad \frac{d t}{d x}=e^{x} \rightarrow d t=e^{x} \cdot d x
$$

$$
\frac{1}{1+\left(e^{x}\right)^{2}} \cdot e^{x}=\frac{e^{x}}{1+e^{2 x}}
$$

$$
\begin{aligned}
& \text { * } \int \ln x d x=\int \ln _{G(x)} \underset{G}{ } \cdot 1 \cdot d x=\ln x \cdot x-\int x(\ln x)^{\prime} d x=x \ln x-\int x \cdot \frac{1}{x} d x=x \ln x-x+c \\
& \text { * } \int x \ln x d x=\frac{x^{2}}{2} \cdot \ln x-\int \frac{x^{2}}{\frac{x^{2}}{2} \ln x} \frac{x}{\frac{x}{2}} \cdot d x=\frac{x^{2}}{2} \ln x-\frac{x^{2}}{4}+c
\end{aligned}
$$

$$
\begin{aligned}
& \text { Pi.nip y"n pk } \\
& \text { betry } \\
& \text { * } \int \ln ^{2} x d x=\int \ln _{G(x)} \ln _{F^{\prime}(x)} x \cdot 1 \cdot d x=x \ln ^{2} x-\int x \underbrace{\left.\ln ^{2} x\right)^{\prime}}_{\frac{2 \ln x}{x}} d x=x \ln ^{2} x-\int 2 \ln x d x=x \ln ^{2} x-2(x \ln x-x) \\
& \text { * } \int \ln \left(x^{2}+1\right) \cdot 1 \cdot d x=x \ln \left(x^{2}+1\right)-\int x \cdot \frac{2 x}{x^{2}+1} d x=x \ln \left(x^{2}+1\right)-2 \int \frac{x^{2}}{x^{2}+1} d x \\
& =x \ln \left(x^{2}+1\right)-2(x-\operatorname{arctg}(x))+c \\
& \text { * } \int_{F^{\prime}(x) G(x) e^{x} \sin x d x}^{e^{\prime}(x)}=e^{x} \sin x-\int_{F^{\prime}(x)} e^{x} \cos x(x) d x=e^{x} \sin x-(\left(e^{x} \cos x\right)-\int e^{x} \underbrace{(\cos x)^{\prime}}_{-\sin x} d x)= \\
& e^{x} \sin x-e^{x} \cos x-\int e^{x} \sin x d x \\
& H(x)=e^{x} \sin x-e^{x} \cos x-H(x) \\
& 2 H(x)=e^{x} \sin x-e^{x} \cos x \\
& H(x)=\frac{e^{x}(\sin x-\cos x)}{2}+c
\end{aligned}
$$

$$
\begin{aligned}
& \text { * } \int \frac{e^{\tan (x)}}{\cos ^{2} x} d x=\int e^{t} \cdot d t=e^{t}=e^{\tan x}+c \\
& \frac{1}{\cos ^{2} x}=\tan ^{\prime}(x) \\
& t=\tan (x) \quad \frac{d t}{d x}=\frac{1}{\cos ^{2} x} \\
& x \int \frac{d x}{x \sqrt{1-\ln ^{2} x}}=\int \frac{d t}{\sqrt{1-t^{2}}}=\arcsin (t)=\arcsin (\ln (x))+c \\
& t=\ln x \\
& \frac{d t}{d x}=\frac{1}{x} \quad d t=\frac{1}{x} \cdot d x=\frac{d x}{x} \\
& \text { * } \left.\int \tan (x) d x=\int \frac{\sin x}{\cos x} d x=-\int \frac{d t}{t}=-\ln |+1=-\ln | \cos x \right\rvert\,+c \\
& t=\cos x \\
& \begin{array}{ll}
\frac{d t}{d x}=-\sin x \quad & d t=-\sin x d x \\
\sin x d x=-d t
\end{array} \\
& \int \frac{x^{4}}{x^{5}-8} d x=\frac{1}{5} \ln \left|x^{5}-8\right| \\
& \int f(x) d x=\int f(g(t)) g^{\prime}(t) d t=H(t)=\begin{array}{c}
\text { nisfr } \\
\text { nifer } \\
\text { nipn }
\end{array} \\
& x=g(t) \\
& d x=g^{\prime}(t) \cdot d t \\
& \text { * } \int \frac{d x}{\sqrt{x}(1+\sqrt[3]{x})}=\frac{6 t^{5} d t}{t^{3}\left(1+t^{2}\right)}=\int \frac{6 t^{2}}{1+t^{2}} d t=6(t-\arctan (t))=6(\sqrt[6]{x}-\operatorname{arctg}(\sqrt[6]{x}))+c \\
& \begin{array}{l}
x=t^{6} \\
d x=6 t^{5} d t \quad * \frac{t^{2}}{t^{2}-1}=\frac{t^{2}+1-1}{t^{2}+1}=1-\frac{1}{t^{2}+1}
\end{array} \\
& \text { * } \int \sqrt{1-x^{2}} d x=\int \cos t \cdot \cos t d t=\int \cos ^{2} t d t=\underbrace{\frac{t}{2}+\frac{\sin 2 t}{4}+c}_{\text {roc } 0 \cdot k} \\
& \cos ^{2} t+\sin ^{2} t=1 \\
& x=\sin t \rightarrow d x=(\sin t)^{\prime} d t=\cos t d t \\
& t=\arcsin (x) \\
& \cos ^{2} t=\frac{1+\cos 2 t}{2} \\
& \sqrt{1-x^{2}}=\sqrt{1-\sin ^{2} t}=\sqrt{\cos ^{2} t}=\cos t
\end{aligned}
$$

$$
=\frac{\arcsin x}{2}+\frac{\sin (2 \arcsin (x))}{4}+c=\frac{\arcsin x}{2}+\frac{x \sqrt{1-x^{2}}}{2}+c
$$

$=\sin (2 \arcsin (x))$ nle D'Ceor $j \%$ $\sin (2 t)=2 \sin t \frac{\sqrt{1-x^{2}}}{\cos t}=$

$$
\begin{aligned}
& \text { * } \int \frac{x^{2} d x}{\sqrt{4-x^{2}}}=\int \frac{4 \sin ^{2} t 2 \cos t d t}{2 \phi \operatorname{sis} t}=4 \int \sin ^{2} t d t=4\left(\frac{t}{2}-\frac{\sin 2 t}{4}\right)+c=2 t-\sin 2 t+c \\
& x=2 \sin t \\
& \sqrt{4-x^{2}}=\sqrt{4-4 \sin ^{2} t}=2 \sqrt{1-\sin ^{2} t}=2 \cos t \\
& \cos t=\sqrt{1-\sin ^{2} t} \quad \sin t=\frac{x}{2} \rightarrow t=\arcsin \left(\frac{x}{2}\right) \rightarrow \sin 2 t=2 \sin t \cos t=x \sqrt{1-\frac{x^{2}}{4}} \\
& 2 \arcsin \left(\frac{x}{2}\right)-x \sqrt{1-\frac{x^{2}}{4}}+c \\
& \text { : nioto niles }
\end{aligned}
$$


\end{document}

% This LaTeX document needs to be compiled with XeLaTeX.
\documentclass[10pt]{article}
\usepackage[utf8]{inputenc}
\usepackage{ucharclasses}
\usepackage{graphicx}
\usepackage[export]{adjustbox}
\graphicspath{ {./images/} }
\usepackage{amsmath}
\usepackage{amsfonts}
\usepackage{amssymb}
\usepackage[version=4]{mhchem}
\usepackage{stmaryrd}
\usepackage{bbold}
\usepackage{polyglossia}
\usepackage{fontspec}
\setmainlanguage{hindi}
\setotherlanguages{thai, english}
\IfFontExistsTF{Noto Serif Devanagari}
{\newfontfamily\hindifont{Noto Serif Devanagari}}
{\IfFontExistsTF{Kohinoor Devanagari}
  {\newfontfamily\hindifont{Kohinoor Devanagari}}
  {\IfFontExistsTF{Devanagari MT}
    {\newfontfamily\hindifont{Devanagari MT}}
    {\IfFontExistsTF{Lohit Devanagari}
      {\newfontfamily\hindifont{Lohit Devanagari}}
      {\IfFontExistsTF{FreeSerif}
        {\newfontfamily\hindifont{FreeSerif}}
        {\newfontfamily\hindifont{Arial Unicode MS}}
}}}}
\IfFontExistsTF{Noto Serif Thai}
{\newfontfamily\thaifont{Noto Serif Thai}}
{\IfFontExistsTF{Thonburi}
  {\newfontfamily\thaifont{Thonburi}}
  {\IfFontExistsTF{FreeSerif}
    {\newfontfamily\thaifont{FreeSerif}}
    {\IfFontExistsTF{Tahoma}
      {\newfontfamily\thaifont{Tahoma}}
      {\newfontfamily\thaifont{Arial Unicode MS}}
}}}
\IfFontExistsTF{CMU Serif}
{\newfontfamily\lgcfont{CMU Serif}}
{\IfFontExistsTF{DejaVu Sans}
  {\newfontfamily\lgcfont{DejaVu Sans}}
  {\newfontfamily\lgcfont{Georgia}}
}
\setDefaultTransitions{\lgcfont}{}
\setTransitionsForDevanagari{\hindifont}{\rmfamily}
\setTransitionsFor{Thai}{\thaifont}{\lgcfont}

\begin{document}
\begin{center}
\includegraphics[max width=\textwidth]{2025_04_04_df6c8911156ae2d40607g-1}
\end{center}

$$
\frac{3 x-4}{7 x^{2}-9}, \frac{x^{7}-8}{x^{3}+x^{2}+x+1} \quad: \text { ICN\&12 }
$$

$2 \geqslant$ गfore pijhio $Q(x)$ fere $\int \frac{P(x)}{Q(x)} d x$ ficce noe verl

$$
\begin{aligned}
& \begin{array}{l}
: l p o \frac{p(x)}{a x^{-b}}=\hat{p}(x)+\frac{d}{a x+b} \\
d x+\int \frac{d}{a x^{2} b}=\int \hat{p}(x) d x+\frac{d}{a} \ln |a x-b|-c
\end{array} \\
& \int \frac{x^{3}+7 x}{2 x+1} d x \quad: \text { 213 } \\
& \frac{x^{3}+7 x}{2 x+1}=\frac{x^{2}}{2}-\frac{x}{4}+\frac{29}{8}-\frac{29 / 8}{2 x+1} \text { : : (1) S } 10 \text { pir.n }
\end{aligned}
$$

$$
\begin{aligned}
& d=1 \text {-e nuj } d x^{2}+a x^{\perp}+b \\
& \int \frac{x^{7}+4}{3 x^{2}+x+6} d x=\frac{1}{3} \int \frac{x^{7}-4}{x^{2}+\frac{x}{3}+2} \\
& : 213
\end{aligned}
$$

$b=\frac{a^{2}}{4},\left(x+\frac{a}{2}\right)^{2}$ : (fia) ense e.) pre fip's (c) $x^{2}+a x+b$ $x=t-\frac{a}{2}, t=x-\frac{a}{2} \quad$ :ros as nาpua

$$
\begin{aligned}
& \int \frac{d_{1} x+d_{2} x}{x^{2}+a x+b}=\int \frac{d_{1}\left(t-\frac{a}{2}\right)+d_{2}}{t^{2}}=\int \frac{d_{1}}{t}-\frac{\frac{a}{2} d_{1}+d_{2}}{t^{2}} d t=d_{1} \ln |t|+\frac{\frac{d_{1} a}{2}-d_{2}}{t}=d_{1} \ln \left(x+\frac{a}{2}\right)+\frac{\frac{d_{1} a}{2}-d_{2}}{x+\frac{a}{2}} \\
& \int \frac{3 x-7}{x^{2}+4 x-4} d x \\
& \text { - Tonclis } \\
& \text { - } \\
& \int \frac{3(t-2)-7}{t^{2}} d t=\int \frac{3}{t} d t-\int \frac{13}{t^{2}} d t=3 \ln |t|-\frac{13}{t}=3 \ln |x-2|-\frac{13}{x+2}+c \Leftrightarrow\left\{\begin{array}{r}
x^{2}-4 x-4=(x-2)^{2} \\
t=x+2 \\
x=t-2
\end{array} d x=d t\right.
\end{aligned}
$$

A, b pon' perers p.rit, $\left(x-x_{1}\right)\left(x-x_{2}\right)$ : pule peerr pen'e 2 for $x^{2}+a x+b$ (2)

$$
\int \frac{d_{1} x+d_{2}}{x^{2}+a x-b} d x=A \ln \left|x-x_{1}\right|-B \ln \left|x-x_{2}\right|+c \quad: \text { sic) } \quad \frac{d_{1} x-d_{2}}{x^{2}+a x+b}=\frac{A}{x-x_{1}}+\frac{B}{x-x_{2}}-e \text { po }
$$

\begin{center}
\includegraphics[max width=\textwidth]{2025_04_04_df6c8911156ae2d40607g-2}
\end{center}

$$
\begin{aligned}
& \frac{3 x-8}{x^{2}+5 x+4}=\frac{A}{x^{2}-1}+\frac{B}{x+4}=\frac{A(x+4)+B(x-1)}{x^{2}+5 x+4} \Rightarrow \begin{array}{l}
A(x+4)-B(x+1)=3 x+8 \\
A x+4 A+B x+B=3 x+8
\end{array} \\
& A+B=3 \quad 4 A+B=8 \\
& B=3-A \quad 4 A+3-A=8 \\
& B=\frac{4}{3} \quad 3 A=5 \quad A=\frac{5}{3} \\
& \underbrace{\frac{3 x 8}{x^{2}+5 x+4}}_{x_{1}=-1 x_{2}=-4}=\frac{5}{3} \ln |x+1|+\frac{4}{3} \ln |x+4|+c \\
& \int \frac{3 x+1}{x^{2}+6 x+11} d x \stackrel{*}{=} \int \frac{3 x+1}{(x-3)^{2}+2} \stackrel{*}{=} \frac{3(t-3)+1}{t^{2}+2} d t=\int \frac{3 t-8}{t^{2}+2}=\frac{3}{2} \ln \left(t^{2}+2\right)-\int \frac{8}{t^{2}+2} d t \\
& :(\mathrm{CrCl} / 3 \\
& =\frac{3}{2} \ln \left(x^{2}+6 x+11\right)-\frac{8}{\sqrt{2}} \arctan \left(\frac{x+3}{\sqrt{2}}\right)+c
\end{aligned}
$$

$$
\begin{aligned}
& t=x+3 \text { : jno j** } \\
& x=t-3 \\
& \int \frac{d t}{t^{2+1}}=\arctan (t)-c \\
& \int \frac{d t}{t^{2}+a^{2}} \overline{=} \frac{1}{a} \arctan \left(\frac{t}{a}\right)^{2} c \\
& {\left[\ln \left(t^{2}+2\right)\right]^{\prime}=\frac{2 t}{t^{2}+2}} \\
& \int \frac{x^{4}}{x^{2}-2 x+5} d x: \text { दcr>13 } \\
& \frac{x^{4}}{x^{2}-2 x+5}=x^{2}+\frac{2 x^{3}+5 x^{2}}{x^{2}+2 x+5}=x^{2}-2 x-\frac{x^{2}-10 x}{x^{2}+2 x+5}=x^{2}-2 x-1+\frac{12 x+5}{x^{2}+2 x+5} \text { : p.4i)/斤分 pl/.n } \\
& \int \frac{x^{4}}{x^{2}+2 x+5} d x=\frac{x^{3}}{3}-x^{2}-x+\int \frac{12 x+5}{x^{2}+2 x+5} \Rightarrow x^{2}+2 x+5=(x-1)^{2}+4 \Rightarrow t=x+1 d t=d x \\
& =\int \frac{12 t-7}{t^{2}+4} d t=6 \ln \left|t^{2}+4\right|-\int \frac{7}{t^{2}+4} d t=6 \ln \left|t^{2}+4\right|-\frac{7}{2} \operatorname{arctg}\left(\frac{t}{2}\right) \\
& \Rightarrow \frac{x^{3}}{3}-x^{2}-x+6 \ln \left(x^{2}+2 x+5\right)-\frac{7}{2} \operatorname{arctg}\left(\frac{x+1}{2}\right)+c \\
& \int \frac{x^{4}}{x^{2}+2 x-3} d x \\
& \text { : } \mathrm{CN} \text { とll } \\
& \int \frac{x^{4}}{x^{2}-2 x-3}=\int x^{2}-2 x-7-\frac{20 x-21}{x^{2}+2 x-3}=\frac{x^{3}}{3}-x^{2}+7 x-\int \frac{20 x-21}{x^{2}+2 x-3} d x \quad \text { : ค. nijso side } \\
& \int \frac{20 x-21}{x^{2}-2 x-3} d x=\int \frac{20 x-21}{(x-1)(x+3)} d x \Rightarrow \frac{20 x-21}{(x-1)(x+3)}=\frac{A}{x-1}+\frac{B}{x+3} \Rightarrow 20 x-21=A x+3 A+B x-B \\
& A+B=20 \quad 3 A-B=-21 \\
& B=20-A \quad 3 A-20+A=-21 \\
& B=\frac{81}{4} \quad 4 A=-1 \\
& \int \frac{x^{4}}{x^{2}-2 x-3}=\frac{x^{3}}{3}-x^{2}+7 x+\frac{1}{4} \ln |x-1|-\frac{81}{4} \ln |x+3|+c
\end{aligned}
$$

$$
\begin{aligned}
& a, b \in \mathbb{R} \\
& \int_{a}^{b} f(x) d x=700 N \\
& \underset{a}{\frac{2}{3} a+\frac{b}{3}} \underset{s_{2}}{s_{3} a+\frac{2}{3} b} b x
\end{aligned}
$$

$$
\begin{aligned}
& \int_{a}^{b} f(x) d x
\end{aligned}
$$


\end{document}

% This LaTeX document needs to be compiled with XeLaTeX.
\documentclass[10pt]{article}
\usepackage[utf8]{inputenc}
\usepackage{ucharclasses}
\usepackage{amsmath}
\usepackage{amsfonts}
\usepackage{amssymb}
\usepackage[version=4]{mhchem}
\usepackage{stmaryrd}
\usepackage{graphicx}
\usepackage[export]{adjustbox}
\graphicspath{ {./images/} }
\usepackage{polyglossia}
\usepackage{fontspec}
\setmainlanguage{thai}
\setotherlanguages{english}
\IfFontExistsTF{Noto Serif Thai}
{\newfontfamily\thaifont{Noto Serif Thai}}
{\IfFontExistsTF{Thonburi}
  {\newfontfamily\thaifont{Thonburi}}
  {\IfFontExistsTF{FreeSerif}
    {\newfontfamily\thaifont{FreeSerif}}
    {\IfFontExistsTF{Tahoma}
      {\newfontfamily\thaifont{Tahoma}}
      {\newfontfamily\thaifont{Arial Unicode MS}}
}}}
\IfFontExistsTF{CMU Serif}
{\newfontfamily\lgcfont{CMU Serif}}
{\IfFontExistsTF{DejaVu Sans}
  {\newfontfamily\lgcfont{DejaVu Sans}}
  {\newfontfamily\lgcfont{Georgia}}
}
\setDefaultTransitions{\lgcfont}{}
\setTransitionsFor{Thai}{\thaifont}{\lgcfont}

\begin{document}
$2|4| 2025$\\
\includegraphics[max width=\textwidth, center]{2025_04_04_b0bfbb17d3d83836590bg-1(5)}\\
\includegraphics[max width=\textwidth, center]{2025_04_04_b0bfbb17d3d83836590bg-1}\\
\includegraphics[max width=\textwidth, center]{2025_04_04_b0bfbb17d3d83836590bg-1(3)}\\
\includegraphics[max width=\textwidth, center]{2025_04_04_b0bfbb17d3d83836590bg-1(2)}\\
\includegraphics[max width=\textwidth, center]{2025_04_04_b0bfbb17d3d83836590bg-1(4)}\\
yาpre

$$
\begin{gathered}
\text { (f } \sqrt{l} \text { inizp } F \text { ) } F^{\prime}=f \begin{array}{c}
:\left(c^{\prime \prime} 12 n n ~\right. \\
p k \\
b \\
s k
\end{array}
\end{gathered}
$$

$$
\int_{a}^{b} f(x) d x=F(b)-F(a)
$$

$$
\int_{0}^{1} x d x=\frac{1^{2}}{2}-\frac{0^{2}}{2}=1=\frac{1^{2}}{2}+3-\frac{0^{2}}{2}-3=1: \operatorname{lcN} \text { र } 13
$$

guinst if lal piona hiscjiky e3NC3. rinn , $F-\lambda$\\
she a, b rapp ripire in fele ** $\int_{a}^{b}$ hre fargjikn er $\int_{a}^{b} f(x) d x \leq 0 \operatorname{sic}[a, b] \operatorname{Sop} f \leqslant 0$ pk

$$
F(b)-F(a)=\int_{a}^{b} F(x)
$$

(1) $\int_{1}^{2} 3 d x=\left.\right|_{2} ^{2} 3 x=3 \cdot 2-3 \cdot 1=3 \xrightarrow[1]{\substack{\underbrace{y}_{2} \\ 3_{2}}} x$\\
(2) $\int_{1}^{2}-3 d x=\left.\right|_{1} ^{2}-3 \cdot 2-(-3 \cdot 1)=-3$\\
(3) $\int_{1}^{2} x d x=\left.\right|_{1} ^{2} \frac{x^{2}}{2}=\frac{4}{2}-\frac{1}{2}=1.5 \underbrace{y}_{1} x$\\
(4) $\int_{0}^{1} x e^{-x} d x=\left.\right|_{0} ^{1}-x \cdot e^{-x}-\int_{0}^{1}-e^{-x}=-1 \cdot e^{-1}+\int_{0}^{1} e^{-x} d x=-\frac{1}{e}-\int_{0}^{1} e^{-x} d x=\frac{1}{e}+\left.\right|_{0} ^{1}-\left(e^{-x}\right)=\frac{1}{e}+\left(-\frac{1}{e}-1\right)=1-\frac{2}{e}>0$\\
\includegraphics[max width=\textwidth, center]{2025_04_04_b0bfbb17d3d83836590bg-1(1)}\\
(5) $\int_{1}^{1} \frac{\ln ^{3} x}{x} d x=\int_{\ln 1}^{\operatorname{tn}} t^{3} d t=\int_{0}^{1} t^{3} d t=\left.\right|_{0} ^{1} \frac{t^{4}}{4}=\frac{1}{4}$

$$
\int_{a}^{b} f^{\prime}(g(x)) g^{\prime}(x) d x=\int_{g(a)}^{g^{(b)}} f(t) d t
$$

: $3 \sqrt{0} 0$

$$
\begin{aligned}
& * \int_{a}^{b} f(x) \pm g(x) d x=\int_{a}^{b} f(x) d x \pm \int_{a}^{b} g(x) d x \\
& * \int_{a}^{b} c \cdot f(x) d x=c \cdot \int_{a}^{b} f(x) d x
\end{aligned}
$$

$$
\int_{a}^{b} f(x) d x \geqslant \int_{a}^{b} g(x) d x
$$

\begin{itemize}
  \item 
\end{itemize}

$$
\begin{aligned}
& * \int_{a}^{a} f(x) d x=0 \\
& * \int_{a}^{b} f(x) d x=-\int_{b}^{a} f(x) \\
& * \int_{a}^{b} f(x) d x+\int_{b}^{c} f(x) d x=\int_{a}^{c} f(x) d x \\
& \left.* \int_{a}^{b} f(x) d x\left|\leqslant \int_{a}^{b}\right| f(x) \mid d x \quad=e \operatorname{Src}\right)^{\prime}(c) \text { efiern } \quad \text { e" } \mid c \\
& \int_{-1}^{n}|x| d x=\int_{0}^{-1} x d x+\int_{-1}^{0}-x d x=\left.\right|_{0} ^{\frac{x^{2}}{2}+\left.\right|_{-1}}-\frac{x^{2}}{2}=\frac{1}{2}-0+\left(0-\left(\left.-\frac{1}{2} \right\rvert\,\right)=1\right.
\end{aligned}
$$

\begin{center}
\includegraphics[max width=\textwidth]{2025_04_04_b0bfbb17d3d83836590bg-2}
\end{center}

$$
x \text { for } f(x)=f(-x) \text { in'eis n'3pノ10 }
$$

$$
x \text { ff } f(x)=-f(-x) \text { :N'ds ic ispرlo }
$$

$$
\begin{aligned}
& \int_{-a}^{a} f(x) d x=2 \cdot \int_{0}^{a} f(x) d x \\
& \int_{-a}^{a} f(x) d x=0
\end{aligned}
$$

$$
\begin{aligned}
& \frac{d t}{d x}=-1 \quad t=-x \quad \int_{-a}^{0} f(x) d x=\int_{a}^{0}-f(t) d t=-\int_{a}^{0} f(t) d t=\int_{0}^{a} f(t) d t \\
& d t=-d x \quad t^{\prime}=-1 \\
& \Rightarrow \int_{-a}^{a} f(x)=\int_{-a}^{0} f(x) d x+\int_{0}^{a} f(x) d x=2 \int_{0}^{a} f(x) d x \\
& \int_{-a}^{0} f(x) d x=\int_{a}^{0} f(-t) \cdot(-d t)=\int_{a}^{0}-f(t) \cdot-d t=-\int_{0}^{a} f(t) d t=-\int_{0}^{a} f(x) d x \text { sicis ic } f .2
\end{aligned}
$$

$$
\begin{aligned}
& \text { * } \left.\int_{a}^{b}(f(x)-g(x)) d x=g-5 \text { f } ; \text { (-15 }\right) \text { neen } \\
& \text { * } \int_{a}^{b}|f(x)-g(x)| d x=\int_{a_{3}}^{a_{1}}(g(x)-f(x)) d x+\int_{a_{1}}^{a_{2}} f(x)-g(x) d x \\
& +\int_{a_{2}}^{a_{3}} g(x)-f(x) d x+\int_{a_{3}}^{b} f(x)-g(x) d x
\end{aligned}
$$

:P.nCe inlein\\
\includegraphics[max width=\textwidth, center]{2025_04_04_b0bfbb17d3d83836590bg-3(1)}\\
\includegraphics[max width=\textwidth]{2025_04_04_b0bfbb17d3d83836590bg-3(3)} x

$$
\int_{0}^{3} 3 x-x^{2} d x=\left.\right|_{0} ^{3} \frac{3}{2} x^{2}-\frac{x^{3}}{3}=\frac{3}{2} \cdot 9-\frac{3^{3}}{3}=\frac{9}{2}
$$

$$
x=0, x=3
$$

\includegraphics[max width=\textwidth, center]{2025_04_04_b0bfbb17d3d83836590bg-3}\\
\includegraphics[max width=\textwidth]{2025_04_04_b0bfbb17d3d83836590bg-3(2)} $y=x^{2}-2 x-8$

$$
S_{2}=\int_{0}^{2}\left(0-\left(x^{2}-2 x-8\right)\right) d x=\int_{0}^{2}-x^{2}+2 x+8 d x=\int_{0}^{2}-\frac{x^{2}}{3}-8 x+x^{2}=16^{2}-4-\frac{8}{3}=20-\frac{8}{3}
$$

$$
\left.s_{1}=\int_{2}^{x_{0}}\left(4-x^{2}\right)-\left(x^{2}-2 x-8\right)\right) d x=\int_{2}^{3}-2 x^{2}+2 x+12=
$$

$$
\begin{array}{r}
\left.\right|_{2} ^{3} x^{2}+12 x-\frac{2}{3} x^{3}=9-4-\frac{2}{3} \cdot 1 \\
s_{1}+s_{2}=37-\frac{46}{3}=21 \frac{2}{3}
\end{array}
$$

$: x_{0}$ ric k $k n j$

$$
\begin{aligned}
& 4-x^{2}=x^{2}-2 x-8 \\
& 2 x^{2}-2 x-12=0 \\
& x^{2}-x-6=0 \\
& x=-2, x=3
\end{aligned}
$$


\end{document}

\documentclass[10pt]{article}
\usepackage[utf8]{inputenc}
\usepackage[T1]{fontenc}
\usepackage{amsmath}
\usepackage{amsfonts}
\usepackage{amssymb}
\usepackage[version=4]{mhchem}
\usepackage{stmaryrd}
\usepackage{graphicx}
\usepackage[export]{adjustbox}
\graphicspath{ {./images/} }

\begin{document}
$$
\begin{aligned}
& 99 \\
& -495 / 5 \\
& -45 \downarrow \\
& \frac{45}{-45} \\
& -\frac{5}{0}
\end{aligned}
$$

: $\operatorname{tary3}$\\
$: 0 \times 100 \quad 5 \mid \mathrm{C} 77$

$$
\begin{gathered}
\frac{x^{3}+4 x^{2}-2 x+6}{x^{4}+3 x^{3}-2 x^{2}+4 x+5} x-1 \\
\frac{x^{4}-x^{3}}{-x^{3}-2 x^{2}+4 x+5} \\
\frac{4 x^{3}-4 x^{2}}{-2 x^{2}+4 x+5} \\
\frac{2 x^{2}-2 x}{11}
\end{gathered}
$$

$$
(x-1)\left(x^{3}+4 x^{2}+2 x+6\right)+11=x^{4}+3 x^{3}-2 x^{2}+4 x+5
$$

$$
\begin{aligned}
& \lim _{x \rightarrow 1} \frac{x^{3}+2 x^{2}-10 x-7}{x-1}=\lim _{x \rightarrow 1} \frac{(x-1)\left(x^{2}+3 x-7\right)}{(x-1)} \quad \frac{x^{2}+3 x-7}{x^{3}+2 x^{2}-10 x+7 \mid x-1}: x^{x^{3}-x^{2}} \\
& =-3
\end{aligned}
$$

\includegraphics[max width=\textwidth, center]{2025_04_04_18550a5b6924dd4bf44fg-2(2)}\\
\includegraphics[max width=\textwidth, center]{2025_04_04_18550a5b6924dd4bf44fg-2}\\
\includegraphics[max width=\textwidth, center]{2025_04_04_18550a5b6924dd4bf44fg-2(1)}

$$
\begin{aligned}
& A=\sin ^{-1}(x+h) \quad B=\sin ^{-1}(x)
\end{aligned}
$$

$$
\begin{aligned}
& \sin A=x+h \quad \sin B=x \\
& \begin{array}{c}
\sin A=x+h \quad \sin B=x \quad \sin x-\sin y=2 \sin \left(\frac{x-y}{2}\right) \cdot \cos \left(\frac{x-y}{2}\right) \\
\sin A-\sin B=h \quad, \quad x 1
\end{array} \\
& \begin{array}{l}
\lim _{h \rightarrow 0} \frac{\arcsin (x+h)-\arcsin (x)}{h}=\lim _{A \rightarrow B} \frac{A-B}{\sin A-\sin B}=\lim _{A \rightarrow B} \frac{\left(\frac{A-B}{2}\right) \cdot \frac{1}{\sin \left(\frac{A-B}{2}\right)} \cdot \cos \left(\frac{A+B}{2}\right)}{\lim _{t \rightarrow 0} \frac{t}{\sin t}=1} \\
=\frac{1}{\cos (B)}=\frac{1}{\sqrt{1-\sin ^{2} B}}=\frac{1}{\sqrt{1-x^{2}}}
\end{array} \\
& \cos ^{-1} x+\sin ^{-1} x=\frac{\pi}{2} \\
& \cos ^{-1} x=\frac{\pi}{2}-\sin ^{-1} x \\
& \begin{array}{ccc}
: \tan ^{-1} x & \sqrt{2} \text { siscs } & \text { anosis } \\
-\frac{y}{+x y} & \rightarrow 1 & \text { :nins }
\end{array} \\
& \tan ^{-1} x-\tan ^{-1} y=\tan ^{-1}\left(\frac{x-y}{1+x y}\right) \\
& \begin{array}{l}
\lim _{h \rightarrow 0} \frac{\tan ^{-1}(x-h)-\tan ^{-1}(x)}{h}=\lim _{h \rightarrow 0} \frac{\tan ^{-1}\left(\frac{h}{1+(x+h) x}\right)}{h}=\lim _{h \rightarrow 0} \frac{\tan ^{-1}(1+(x+h) x}{h} \\
=\frac{1}{1+x^{2}}
\end{array} \\
& (f(g(x)))^{\prime}=f^{\prime}\left(g(x) \cdot g^{\prime}(x) \quad\right. \text { :л⿰七刀口 } \\
& \sin ^{-1}\left(x^{2}\right)=\frac{1}{\sqrt{1-\left(x^{2}\right)^{2}}} \cdot 2 x \\
& f^{\prime}(x)=2 x \cdot \tan ^{-1} x+x^{2} \cdot \frac{1}{1+x^{2}} \\
& \left.f(x)=x^{2} \tan ^{-1} x \quad \text {-s } \Omega>\varepsilon\right) \text { ipen } \\
& \left.\cos ^{-1}\left(e^{-x^{2}}\right)\right)^{\prime}=\frac{1}{\sqrt{1-e^{-2 x^{2}}}} \cdot 2 x \cdot e^{-x^{2}}
\end{aligned}
$$

$$
\begin{aligned}
& \left(\sin ^{-1}\left(\frac{x}{\sqrt{1+x^{2}}}\right)\right)^{\prime}=\frac{1}{\left.\sqrt{1-\left(\frac{x}{\sqrt{1+x^{2}}}\right.}\right)^{2}} \cdot \frac{1\left(\sqrt{1+x^{2}}\right)-\frac{1 \cdot x}{2 \sqrt{1+x^{2}}}}{1+x^{2}}=\frac{1}{\sqrt{\frac{1+x^{2}-x^{2}}{1+x^{2}}}} \cdot \frac{1+x^{2}-x^{2}}{\left(1+x^{2}\right)^{1 \cdot 5}}=\frac{1}{1+x^{2}} \\
& \lim _{x \rightarrow 1^{-}} \frac{\sin ^{-1}(x)-\frac{\pi}{2}}{\sqrt{1-x}} \stackrel{L}{ }=\lim _{x \rightarrow 1^{-}} \frac{1}{\sqrt{1-x^{2}}}=\lim _{x \rightarrow 1^{-}} \frac{-2 \sqrt{1-x}}{\sqrt{(1-x)(1+x)}}=\lim _{x \rightarrow 1^{-}} \frac{2 \cdot \sqrt{1-x} \sqrt{1-x} \sqrt{1+x}}{\sqrt{1-x}}=\frac{-2}{\sqrt{2}}=-\sqrt{2} \\
& h(x)+\cos ^{-1}(h(x))=x \\
& h^{\prime}(x)-\frac{1}{\sqrt{n-(h(x))^{2}}} \cdot h^{\prime}(x)=1: \lim _{x \rightarrow n} h^{\prime}(x) \text { alcilen fk nue) } \\
& h^{\prime}(x)\left[1-\frac{1}{\sqrt{1-(h(x))^{2}}}\right]=1 \\
& h^{\prime}(x)=\frac{1}{1-\frac{1}{\sqrt{1-(h(x))^{2}}}} \\
& \lim _{x \rightarrow 1^{-}} h^{\prime}(x)=\lim _{x \rightarrow 1^{-}} \frac{1}{1-\frac{1}{\sqrt{1-h(x)^{2}}}} \\
& \lim _{x \rightarrow 0} \frac{\tan ^{-1}(x)-x}{x^{3}} \stackrel{L}{=} \lim _{x \rightarrow 0} \frac{\frac{1}{1+x^{2}}-1}{x^{2}}=\lim _{x \rightarrow 0} \frac{1-\left(1+x^{2}\right)}{\left(1-x^{2}\right) \cdot 3 x^{2}}=\lim _{x \rightarrow 0} \frac{-x^{2}}{3 x^{2}+3 x^{4}}=\frac{-1}{3+3 x^{2}}=-\frac{1}{3} \\
& \lim _{x \rightarrow 0} \frac{\sin ^{-1} x+\cos ^{-1} x-\frac{\pi}{2}}{x^{2}} \stackrel{\lim _{x \rightarrow 0}}{ } \frac{\frac{1}{\sqrt{1+x^{2}}}-\frac{1}{\sqrt{1+x^{2}}}}{2 x}=\lim _{x \rightarrow 0} \frac{0}{2 x}=0
\end{aligned}
$$


\end{document}

\documentclass[10pt]{article}
\usepackage[utf8]{inputenc}
\usepackage[T1]{fontenc}
\usepackage{amsmath}
\usepackage{amsfonts}
\usepackage{amssymb}
\usepackage[version=4]{mhchem}
\usepackage{stmaryrd}
\usepackage{graphicx}
\usepackage[export]{adjustbox}
\graphicspath{ {./images/} }

\begin{document}
2713125

$$
\begin{aligned}
& F(x)=\int f(x) d x \\
& \left(e^{x}\right)^{\prime}=e^{x} \rightarrow \int e^{x}=e^{x}+c \\
& (\ln x)^{\prime}=\frac{1}{x} \rightarrow \int \frac{1}{x} d x=\ln x+c
\end{aligned}
$$

\begin{itemize}
  \item $\int \frac{(x-2)^{2}}{x} d x=\int \frac{x^{2}-4 x+4}{x} d x=\int x-4+\frac{4}{x} d x=\frac{x^{2}}{2}-4 x+4 \cdot \ln |x|+c$
\end{itemize}

$$
\Rightarrow \int e^{-3 x} d x=\frac{e^{-3 x}}{-3}+c
$$

\begin{itemize}
  \item $\int \frac{3^{2 x}+5^{x-1}}{4^{x}} d x=\int \frac{9^{x}+5 \cdot 5^{x}}{4^{x}} d x=\int\left(\frac{9}{4}\right)^{x}+5 \cdot\left(\frac{5}{4}\right)^{x} d x=\frac{\left(\frac{9}{4}\right)^{x}}{\ln \left(\frac{9}{4}\right)}+\frac{5\left(\frac{5}{4}\right)^{x}}{\ln \left(\frac{5}{4}\right)}$\\
( $\int \frac{3^{2 x}+5^{x+1}}{4^{x}} d x=\int \frac{9^{x}+5 \cdot 5^{x}}{4^{x}} d x=\int\left(\frac{9}{4}\right)^{x}$\\
(1) $\int \frac{d x}{x^{2}-4 x+5}=\int \frac{d x}{(x+2)^{2}+1}=\arctan (x+2)+c$
\end{itemize}

$$
\begin{aligned}
& \text { ( } \int \frac{d x}{x^{2}-4 x+6}=\int \frac{d x}{(x+2)^{2}+2}=\frac{1}{2} \int \frac{d x}{\left(\frac{x+2}{\sqrt{2}}\right)^{2}-1}=\left(\arctan \left(\frac{x+2}{\sqrt{2}}\right) \cdot \frac{1}{\sqrt{2}} \cdot \frac{1}{2}+c\right. \\
& * \int \frac{d x}{\sqrt{x}-\sqrt{x+1}}=\int \frac{\sqrt{x}+\sqrt{x+1}}{(\sqrt{x}-\sqrt{x+1})(\sqrt{x}+\sqrt{x+1})} d x=\int \frac{\sqrt{x}+\sqrt{x+1}}{x-(x+1)} d x=-\left(\frac{x^{1.5}}{1.5}+\frac{(x+1)^{1.5}}{1.5}\right)+c
\end{aligned}
$$

$$
f^{\prime}(x)=\int f^{\prime \prime}(x) d x=-\cos x-\sin x+c
$$

I $x \mid 20 \delta 147)$\\
\includegraphics[max width=\textwidth, center]{2025_04_04_3fe66dedb3c39861a5a7g-1}

$$
f(x)=\int f^{\prime}(x) d x=-\sin x+\cos x+c x+d
$$

$$
0=f^{\prime}(0)=-\cos 0-\sin 0+c \Rightarrow c=1
$$

$$
0=f(0)=-\sin 0+\cos 0+1 \cdot 0+d \Rightarrow d=-1
$$

$$
f^{\prime \prime}(x)=2 \rightarrow f^{\prime}(x)=S f^{\prime \prime}(x) d x=2 x+c
$$

$$
f^{\prime}(x)=2 x+c \rightarrow f(x)=\int f^{\prime}(x) d x=x^{2}+c x+d
$$

$-3=2 \cdot 1+c \quad=$ fore slcilen $c=-5$\\
: SDI pier pinin

$$
\begin{aligned}
& -3 \cdot 1+5=1^{2}-5 \cdot 1+d \\
& 2=-4+d \\
& d=6
\end{aligned}
$$

$$
\begin{aligned}
& \text { * } \int \frac{x+2}{\left(9 x^{2}+42 x+49\right)^{\frac{1}{3}}} d x=\int \frac{x-2}{(3 x+7)^{2 / 3}} d x=\int \frac{\frac{1}{3}(3 x+7-1)}{(3 x-7)^{2 / 3}} d x=\frac{1}{3} \int \frac{(3 x+7)^{1}}{(3 x+7)^{2 / 3}}-(3 x+7)^{-\frac{2}{3}} d x \\
& =\frac{1}{3} \cdot \frac{1}{3}(3 x+7)^{4 / 3}-\frac{1}{3} \cdot \frac{1}{3}(3 x+7)^{1 / 3}+c=\frac{1}{9}(3 x+7)^{1 / 3}[3 x-7-1]+c
\end{aligned}
$$

$$
\begin{aligned}
& \int f^{\prime}(x) \cdot g(x) d x=f(x) g(x)=\int g^{\prime}(x) f(x) d x \\
& \text { * } \int \frac{x}{e^{x}} d x=\int x e^{-x} d x=x \cdot-e^{-x}+\int e^{-x} d x=-x e^{-x}-e^{-x}+c \\
& u=x \quad u^{\prime}=1 \\
& =-e^{-x}(x+1)+c \\
& \text { * } \int x(3 x+1)^{20} d x=\frac{x \cdot(3 x+1)^{21}}{3 \cdot 21}-\int \frac{1 \cdot(3 x+1)^{21}}{63} d x \\
& =\frac{x(3 x+1)^{21}}{63}-\frac{(3 x+1)^{22}}{63 \cdot 22 \cdot 3}+c \\
& v^{\prime}=e^{-x} \quad v=e^{-x} \\
& f=x \quad f^{\prime}=1 \\
& g^{\prime}=(3 x+1)^{20} \\
& g=\frac{(3 x+1)^{21}}{63} \\
& \begin{aligned}
* \int e^{x} \cdot \sin x d x= & e^{x} \cdot(-\cos x)+\int e^{x} \cos x d x=-e^{x} \cos x+e^{x} \sin x-\int e^{x} \sin x d x \quad f=e^{x} \quad f^{\prime}=e^{x} \\
\qquad & \\
& \int e^{x} \sin x d x=-e^{x} \cos x+e^{x} \sin x-\int e^{x} \sin x d x \\
& 2 \int e^{x} \sin x d x=e^{x}(\sin x-\cos x) \\
& \int e^{x} \sin x d x=-\cos x \\
& \frac{e^{x}(\sin x-\cos x)}{2}
\end{aligned} \\
& \int f(x) d x \quad t=g(x) \\
& d t=g^{\prime}(x) d x \\
& d x=\frac{d t}{g^{\prime}(x)} \\
& \text { * } \int \frac{1}{x \cdot \ln ^{6} x} d x=\int \frac{1}{\ln ^{6} x} \cdot \frac{1}{x} d x=\int \frac{1}{t^{6}} d t \\
& t=\ln (x) \\
& =\int t^{-6} d t=\frac{t^{-5}}{-5}+c=\frac{(\ln x)^{-5}}{-5}+c \\
& d t=\frac{1}{x} d x \\
& \text { * } \int \frac{e^{x}}{\sqrt{1-e^{2 x}}} d x=\int \frac{1}{\sqrt{1-e^{2 x}}} \cdot e^{x} d x=\int \frac{1}{\sqrt{1-t^{2}}} d t=\arcsin (t)^{2} c= \\
& =\arcsin \left(e^{x}\right)-c \\
& t=e^{x} \\
& e^{x} d x=d t \\
& d x=\frac{d t}{t} \\
& \text { * } \int \tan 2 x d x=\int \frac{\sin 2 x}{\cos 2 x} d x=\frac{1}{-2} \int \frac{-2 \sin 2 x}{\cos 2 x} d x=-\frac{1}{2} \int \frac{d t}{t} \\
& \cos 2 x=t \\
& =-\frac{1}{2} \ln |t|+c=-\frac{1}{2} \ln |\cos 2 x|+c
\end{aligned}
$$

$$
\begin{aligned}
& \text { * } \int e^{\sqrt{x}} d x=2 \int e^{t} t d t=2 \cdot\left[e^{t} \cdot t-\int e^{t}\right]=2\left(e^{t} \cdot t-e^{t}\right)+c \\
& =2\left(e^{\sqrt{x}} \sqrt{x}-e^{\sqrt{x}}\right)+c=2 e^{\sqrt{x}}(\sqrt{x}-1)+c \\
& \frac{1}{2 \sqrt{x}} d x=d t \\
& t=g \\
& 1=g^{\prime} \\
& f^{\prime}=e^{t} \\
& d x=d t \cdot 2 \sqrt{x} \quad f=e^{t} \\
& =d t \cdot 2 t \\
& \text { * } \int e^{\sqrt{5 x-7}} d x=\frac{1}{5} \int e^{\sqrt{t}} d t=\frac{1}{5}\left[2 e^{\sqrt{t}}(\sqrt{t}-1)\right]+c \\
& =\frac{1}{5}\left[2 e^{\sqrt{5 x-7}}(\sqrt{5 x-7}-1)\right]+c \\
& t=5 x-7 \\
& d t=5 d x \\
& d x=\frac{d t}{5} \\
& \text { * } \int \frac{\ln ^{2} x}{x^{2}} d x=\int \frac{\ln ^{2} x}{x} \cdot \frac{1}{x} d x=\int t^{2} \cdot e^{-t} d t \\
& \ln x=t \quad x=e^{t} \\
& =-e^{-t} \cdot t^{2}-\int 2 t \cdot e^{-t} d t=-e^{-t} \cdot t^{2} \cdot \underset{\text { IPPe'n } 2}{ } t \cdot e^{-t} d t \\
& \frac{1}{x} d x=d t \\
& f=t^{2} \rightarrow f^{\prime}=2 t \\
& =-e^{-t} \cdot t^{2}+2 e^{-t}(t-1)+c \\
& =e^{-t}\left(-t^{2}+2 t+2\right)+c=e^{\ln \frac{1}{x}}\left(-\ln ^{2} x+2 \ln x+2\right)+c \\
& =\frac{1}{x}\left(-\ln ^{2} x+2 \ln x-2\right)+c \\
& h(x)+\cos ^{-1}(h(x))=x \\
& h^{\prime}(x)-\frac{h^{\prime}(x)}{\sqrt{1-h^{2}(x)}}=1 \\
& h^{\prime}(x) \cdot\left[1-\frac{1}{\sqrt{1-h^{2}(x)}}\right]=1 \\
& h(x)=1 \\
& \begin{array}{l}
\text { :1pte fiple nice } \\
\lim _{x \rightarrow 1}^{\prime}(x) \text { sk ic3 } N
\end{array} \\
& h(x)-\cos ^{-1}(h(x))=x \\
& \cos ^{-1}(h(x))=0 \\
& \lim _{x \rightarrow 1} h^{\prime}(x)=\lim _{x \rightarrow 1} \frac{1}{1-\frac{1}{\sqrt{1-h^{2}(x)}}}=0 \\
& x \rightarrow 1 \quad h(x) \rightarrow 1
\end{aligned}
$$


\end{document}

% This LaTeX document needs to be compiled with XeLaTeX.
\documentclass[10pt]{article}
\usepackage[utf8]{inputenc}
\usepackage{ucharclasses}
\usepackage{graphicx}
\usepackage[export]{adjustbox}
\graphicspath{ {./images/} }
\usepackage{amsmath}
\usepackage{amsfonts}
\usepackage{amssymb}
\usepackage[version=4]{mhchem}
\usepackage{stmaryrd}
\usepackage{bbold}
\usepackage[fallback]{xeCJK}
\usepackage{polyglossia}
\usepackage{fontspec}
\IfFontExistsTF{Noto Serif CJK JP}
{\setCJKmainfont{Noto Serif CJK JP}}
{\IfFontExistsTF{STSong}
  {\setCJKmainfont{STSong}}
  {\IfFontExistsTF{Droid Sans Fallback}
    {\setCJKmainfont{Droid Sans Fallback}}
    {\setCJKmainfont{SimSun}}
}}

\setmainlanguage{thai}
\setotherlanguages{english, hindi, bengali}
\IfFontExistsTF{Noto Serif Thai}
{\newfontfamily\thaifont{Noto Serif Thai}}
{\IfFontExistsTF{Thonburi}
  {\newfontfamily\thaifont{Thonburi}}
  {\IfFontExistsTF{FreeSerif}
    {\newfontfamily\thaifont{FreeSerif}}
    {\IfFontExistsTF{Tahoma}
      {\newfontfamily\thaifont{Tahoma}}
      {\newfontfamily\thaifont{Arial Unicode MS}}
}}}
\IfFontExistsTF{Noto Serif Devanagari}
{\newfontfamily\hindifont{Noto Serif Devanagari}}
{\IfFontExistsTF{Kohinoor Devanagari}
  {\newfontfamily\hindifont{Kohinoor Devanagari}}
  {\IfFontExistsTF{Devanagari MT}
    {\newfontfamily\hindifont{Devanagari MT}}
    {\IfFontExistsTF{Lohit Devanagari}
      {\newfontfamily\hindifont{Lohit Devanagari}}
      {\IfFontExistsTF{FreeSerif}
        {\newfontfamily\hindifont{FreeSerif}}
        {\newfontfamily\hindifont{Arial Unicode MS}}
}}}}
\IfFontExistsTF{Noto Serif Bengali}
{\newfontfamily\bengalifont{Noto Serif Bengali}}
{\IfFontExistsTF{Kohinoor Bangla}
  {\newfontfamily\bengalifont{Kohinoor Bangla}}
  {\IfFontExistsTF{Bangla MN}
    {\newfontfamily\bengalifont{Bangla MN}}
    {\IfFontExistsTF{Lohit Bengali}
      {\newfontfamily\bengalifont{Lohit Bengali}}
      {\IfFontExistsTF{FreeSerif}
        {\newfontfamily\bengalifont{FreeSerif}}
        {\newfontfamily\bengalifont{Arial Unicode MS}}
}}}}
\IfFontExistsTF{CMU Serif}
{\newfontfamily\lgcfont{CMU Serif}}
{\IfFontExistsTF{DejaVu Sans}
  {\newfontfamily\lgcfont{DejaVu Sans}}
  {\newfontfamily\lgcfont{Georgia}}
}
\setDefaultTransitions{\lgcfont}{}
\setTransitionsFor{Thai}{\thaifont}{\lgcfont}
\setTransitionsForDevanagari{\hindifont}{\rmfamily}
\setTransitionsFor{Bengali}{\bengalifont}{\lgcfont}

\begin{document}
\includegraphics[max width=\textwidth]{2025_04_04_80264a2aee4f4aa0aca3g-1(2)} $\forall x \in I \quad F^{\prime}(x)=f(x)$

$$
\longrightarrow F^{\prime}(x)=\left(\frac{x 3}{3}\right)^{\prime}=x^{2}
$$

$$
\frac{x^{3}}{3} n\left(\cdots \delta \pi \delta(x) F(G), x^{2}=f(a) o k\right.
$$

$\frac{x^{3}}{3}+c \quad 16$\\
\includegraphics[max width=\textwidth, center]{2025_04_04_80264a2aee4f4aa0aca3g-1(3)}\\
\includegraphics[max width=\textwidth, center]{2025_04_04_80264a2aee4f4aa0aca3g-1}\\
\includegraphics[max width=\textwidth]{2025_04_04_80264a2aee4f4aa0aca3g-1(1)} ชเลคス গाऽアコ)

$$
\longrightarrow(F(x)-G(x))^{\prime}=f(x)-f(x)=0
$$

ook mill pant xiap าoon Se soocs

$$
F(x)-G(x)=c
$$

$$
F(x)=G(x)+c \text {. x|apa o. ofקa: }
$$

:\\
prebule\\
(1) $\int 1 d x=\int d x$\\
(2) $\int \frac{1}{x^{2}} d x=\int \frac{d x}{x^{2}}$\\
O

$$
\begin{aligned}
\int f(x) d x=\left\{F(x): F^{\prime}(x)=f(x), \forall x\right\} & =\{G(x): G(x)=F(x)+c\} \quad: \text { nาว刀 }! \\
& =F(x)+c
\end{aligned}
$$

$$
\int x d x=\frac{x^{2}}{2}+c, c \in \mathbb{R}
$$

\includegraphics[max width=\textwidth, center]{2025_04_04_80264a2aee4f4aa0aca3g-2(4)}\\
\includegraphics[max width=\textwidth, center]{2025_04_04_80264a2aee4f4aa0aca3g-2(3)}\\
\includegraphics[max width=\textwidth, center]{2025_04_04_80264a2aee4f4aa0aca3g-2(1)}\\
$\left(x^{2}\right)^{\prime}=2 x \quad: 778$ ว\\
\includegraphics[max width=\textwidth, center]{2025_04_04_80264a2aee4f4aa0aca3g-2(2)}

$$
\int 2 x e^{-x^{2}} d x=e^{-x^{2}}+c \quad \text { akd nnixd }
$$

\includegraphics[max width=\textwidth, center]{2025_04_04_80264a2aee4f4aa0aca3g-2(5)}\\
\includegraphics[max width=\textwidth, center]{2025_04_04_80264a2aee4f4aa0aca3g-2}\\
(1) $(\sin (x))^{\prime}=\cos (x)$\\
(2)

$$
\begin{array}{r}
(\ln (x))^{\prime}=1 / x \quad, \quad x>0 \\
\longleftrightarrow \int \frac{d x}{x}=\ln (x)+c
\end{array}
$$

: cip

$$
\longrightarrow \int \cos (x) d x=\sin (x)+c
$$

Di.3nc (J.k riknois\\
(1) $\int x^{\alpha} d x=\frac{x^{\alpha+1}}{\alpha+1}+C, \alpha \neq-1$\\
(2) $\int \sin (x) d x=-\cos (x)+c$\\
(3) $\int \cos (x) d x=\sin (x)+c$\\
(4) $\int a^{x} d x=\frac{a^{x}}{\ln (a)}+c$\\
(5) $\int \frac{d x}{a^{2}+x^{2}}=\frac{\arctan \left(\frac{x}{a}\right)}{a}+C$\\
(6) $\int \frac{\alpha x}{\sqrt{a^{2}-x^{2}}}=\arcsin \left(\frac{x}{a}\right)+c$\\
(7) $\int \frac{1}{x} d x=\ln |x|+c$\\
(8) $\int \frac{d x}{x^{2}}=\frac{-1}{x}+c$\\
(9) $\int \sqrt{x} d x=\frac{x^{3 / 2}}{3 / 2}+c=\frac{2 x^{3 / 2}}{3}+c$\\
\includegraphics[max width=\textwidth, center]{2025_04_04_80264a2aee4f4aa0aca3g-3}\\
\includegraphics[max width=\textwidth, center]{2025_04_04_80264a2aee4f4aa0aca3g-3(1)}

$$
\int(\alpha f(x)+\beta g(x)) d x=\alpha \int f(x) d x+\beta \int g(x) d x
$$

(1) $\int(x-3 \sin (x)) d x=\int x d x-3 \int \sin (x) d x=\frac{x^{2}}{2}+3 \cos (x)+c$\\
(2) $\int \frac{x+1}{x} d x=\int\left(1+\frac{1}{x}\right) d x=\int d x+\int \frac{d x}{x}=x+\ln |x|+c$\\
(3) $\int\left(x+\cos ^{2}\left(\frac{x}{2}\right)\right) d x$

$$
\cos ^{2}(\alpha)=\frac{1-\cos (2 \alpha)}{2}: \text { เN }
$$

$$
\begin{aligned}
=\int x d x+\int \frac{1-\cos (x)}{2} & =\frac{x^{2}}{2}+\frac{1}{2} \int d x-\frac{1}{2} \int \cos (x) d x \\
& =\frac{x^{2}}{2}+\frac{x}{2}-\frac{\sin (x)}{2}+c=\frac{x^{2}+x-\sin (x)}{2}+c
\end{aligned}
$$

(4)\\
(1) $\int \cos (2 x) d x=\frac{\sin (2 x)}{2}+c$\\
(2) $\int \frac{\alpha x}{5 x+7}=\frac{\ln |5 x+7|}{5}+c$\\
(3) $\int 3^{5 x+2} d x=3^{5 x+2} \cdot \frac{1}{5} \cdot \frac{\ln (3)}{}+c=\frac{3^{5 x+2}}{5 \ln (3)}+c$\\
: Mlaien nikncip

$$
\begin{aligned}
& \text { (1) } \left.\int \frac{d x}{a^{2}+x^{2}}=\frac{\arctan \left(\frac{x}{a}\right)}{a}+C\right\} \\
& \text { Do rik nix } \\
& =\frac{1}{a^{2}} \int \frac{d x}{1+\left(\frac{x}{a}\right)^{2}} \quad\left[\int \frac{d x}{1+x^{2}}=\arctan (x)+c\right] \\
& =\frac{1}{a^{2}} \cdot \arctan \left(\frac{x}{a}\right) \cdot \frac{1}{1 / a}+c=\frac{\arctan \left(\frac{x}{a}\right)}{a}+C
\end{aligned}
$$

$$
\begin{aligned}
& \int \frac{x^{4}}{x^{2}+1} d x=\int \frac{x^{4}-1+1}{x^{2}+1} d x=\int \frac{\left(x^{2}+1\right)\left(x^{2}-1\right)+1}{x^{2}+1} d x=\int\left(x^{2}-1+\frac{1}{x^{2}+1}\right) d x \\
& =\frac{x^{3}}{3}-x+\arctan (x)+c
\end{aligned}
$$

$$
\begin{aligned}
& {\left[\frac{F(\alpha x+\beta)}{\alpha}\right]^{\prime}=\frac{1}{\alpha} \cdot f(\alpha x+\beta) \cdot \alpha=f(\alpha x+\beta) \quad:(\text { (ว) } \alpha) \text { คกวกก }}
\end{aligned}
$$

(2)

$$
\begin{aligned}
& \left.a>0, \quad \int \frac{d x}{\sqrt{a^{2}-x^{2}}}=\arcsin \left(\frac{x}{a}\right)+c\right\} \text { ว刀 rk n.11s } \\
& =\int \frac{d x}{\sqrt{a^{2}} \cdot \sqrt{1-\left(\frac{x}{a}\right)^{2}}}=\frac{1}{a} \int \frac{d x}{\sqrt{1-\left(\frac{x}{a}\right)^{2}}} \longleftarrow\left[\int \frac{d x}{\sqrt{1-x^{2}}}=\arcsin (x)+c\right] \\
& =\frac{1}{a} \cdot \arcsin \left(\frac{x}{a}\right) \cdot \frac{1}{1 / a}+c=\arcsin \left(\frac{x}{a}\right)+c
\end{aligned}
$$

oipfona sunclsk

$$
\begin{aligned}
& \longrightarrow[u(x) v(x)]^{\prime}=u^{\prime}(x) v(x)+u(x) v^{\prime}(x) \\
& u^{\prime}(x) v(x)=[u(x) v(x)]^{\prime}-u(x) v^{\prime}(x) \\
& \longrightarrow \int u^{\prime}(x) v(x) d x=\int[u(x) v(x)]^{\prime} d x-\int u(x) v^{\prime}(x) d x \\
&=u(x) v(x)-\int u(x) v^{\prime}(x) d x \\
& \int u^{\prime} v d x=u v-\int u v^{\prime} d x \quad: \text { nilleว o' d'NA }
\end{aligned}
$$

(1) $\int x \ln (x) d x\left[\begin{array}{ll}u^{\prime}=x & v=\ln (x) \\ u=x^{2} / 2 & v^{\prime}=1 / x\end{array}\right] \rightarrow \frac{x^{2} \ln (x)}{2}-\int \frac{x}{2} d x$

$$
=\frac{x^{2} \ln (x)}{2}-\frac{x^{2}}{4}+c
$$

(2) $\int x \cos (x) d x\left[\begin{array}{ll}u^{\prime}=\cos (x) & v=x \\ u=\sin (x) & v^{\prime}=1\end{array}\right] \rightarrow x \sin (x)-\int \sin (x) d x$

$$
=x \sin (x)+\cos (x)+c
$$

(3) $\int \ln (x) d x=\int 1 \cdot \ln (x) d x\left[\begin{array}{ll}u^{\prime}=1 & v=\ln (x) \\ u=x & v^{\prime}=1 / x\end{array}\right] \rightarrow x \ln (x)-\int d x$

$$
=x \ln (x)-x+c
$$

infinite loop $\quad \iint\left[e^{x} \sin (x)\right] d x \quad$ : Ni $\lambda \longleftarrow$


\end{document}

% This LaTeX document needs to be compiled with XeLaTeX.
\documentclass[10pt]{article}
\usepackage[utf8]{inputenc}
\usepackage{ucharclasses}
\usepackage{amsmath}
\usepackage{amsfonts}
\usepackage{amssymb}
\usepackage[version=4]{mhchem}
\usepackage{stmaryrd}
\usepackage{graphicx}
\usepackage[export]{adjustbox}
\graphicspath{ {./images/} }
\usepackage{polyglossia}
\usepackage{fontspec}
\setmainlanguage{hindi}
\setotherlanguages{english, hebrew, thai}
\IfFontExistsTF{Noto Serif Devanagari}
{\newfontfamily\hindifont{Noto Serif Devanagari}}
{\IfFontExistsTF{Kohinoor Devanagari}
  {\newfontfamily\hindifont{Kohinoor Devanagari}}
  {\IfFontExistsTF{Devanagari MT}
    {\newfontfamily\hindifont{Devanagari MT}}
    {\IfFontExistsTF{Lohit Devanagari}
      {\newfontfamily\hindifont{Lohit Devanagari}}
      {\IfFontExistsTF{FreeSerif}
        {\newfontfamily\hindifont{FreeSerif}}
        {\newfontfamily\hindifont{Arial Unicode MS}}
}}}}
\IfFontExistsTF{Noto Serif Hebrew}
{\newfontfamily\hebrewfont{Noto Serif Hebrew}}
{\IfFontExistsTF{Arial Hebrew}
  {\newfontfamily\hebrewfont{Arial Hebrew}}
  {\IfFontExistsTF{FreeSerif}
    {\newfontfamily\hebrewfont{FreeSerif}}
    {\newfontfamily\hebrewfont{David}}
}}
\IfFontExistsTF{Noto Serif Thai}
{\newfontfamily\thaifont{Noto Serif Thai}}
{\IfFontExistsTF{Thonburi}
  {\newfontfamily\thaifont{Thonburi}}
  {\IfFontExistsTF{FreeSerif}
    {\newfontfamily\thaifont{FreeSerif}}
    {\IfFontExistsTF{Tahoma}
      {\newfontfamily\thaifont{Tahoma}}
      {\newfontfamily\thaifont{Arial Unicode MS}}
}}}
\IfFontExistsTF{CMU Serif}
{\newfontfamily\lgcfont{CMU Serif}}
{\IfFontExistsTF{DejaVu Sans}
  {\newfontfamily\lgcfont{DejaVu Sans}}
  {\newfontfamily\lgcfont{Georgia}}
}
\setDefaultTransitions{\lgcfont}{}
\setTransitionsForDevanagari{\hindifont}{\rmfamily}
\setTransitionsFor{Hebrew}{\hebrewfont}{\lgcfont}
\setTransitionsFor{Thai}{\thaifont}{\lgcfont}

\begin{document}
$k \in N \quad a \neq 0 \quad a, p \in R \quad$ reb $\quad \frac{a}{(x-p)^{\kappa}}: 1$ गาр $N$.

$$
\begin{array}{r}
\longrightarrow \int \frac{a}{(x-p)^{k}} d x\left[\begin{array}{ll}
t=x-p & \alpha t=\alpha x
\end{array}\right] \rightarrow \int \frac{a}{t^{k}} d x=a \int t^{-k} d x=\frac{a t^{1-k}}{1-k}+c \\
=\frac{a(x-p)^{1-k}}{1-k}+c \\
\int \frac{2}{(x-5)^{2}} d x \quad[t=x-5 \quad \alpha t=\alpha x] \longrightarrow 2 \int \frac{d x}{t^{2}}=\frac{-2}{t}+c=\frac{-2}{x-5}+c \\
\quad a, b \neq 0 \quad a, b, p, q \in R \quad \frac{a x+b}{x^{2}+p x+q} \quad: 277 \rho N
\end{array}
$$

(k) (k) גכנת סריק\\
(1) $\int \frac{d x}{(x+1)(x+2)}$

$$
\begin{aligned}
& =\int\left(\frac{1}{x+1}-\frac{1}{x+2}\right) d x \\
& =\int \frac{d x}{x+1}-\int \frac{d x}{x+2}
\end{aligned}
$$

$$
\left|\begin{array}{l}
\left(\frac{1}{(x+1)(x+2)}=\frac{a}{x+1}+\frac{b}{x+2}\right)(x+1)(x+2): \operatorname{NC} C \\
1=a(x+2)+b(x+1)=a x+2 a+b x+b \\
1=\underbrace{(a+b) x}_{a+b=0}+\underbrace{(2 a+b)}_{2 a+b=1} \longrightarrow a=1 \quad b=-1
\end{array}\right|
$$

$$
=\ln |x+1|-\ln |x+2|+c=\ln \left|\frac{x+1}{x+2}\right|+c
$$

$$
\begin{aligned}
& \text { (2) } \int \frac{x}{x^{2}+2 x-3} d x=\int \frac{x}{(x+3)(x-1)} d x\left(\frac{x}{(x+3)(x-1)}=\frac{a}{x+3}+\frac{b}{x-1}\right) \cdot(x+3)(x-1) \quad: \quad 7(16) \\
& =\frac{3}{4} \int \frac{d x}{x+3}+\frac{1}{4} \int \frac{d x}{x-1} \quad x=a(x-1)+b(x+3)=a x-a+b x+3 b \\
& =\frac{3 \ln |x+3|}{4}+\frac{\ln |x-1|}{4}+c \\
& \begin{array}{l}
x=a(x-1)+b(x+3)=a x-a+b x+3 b \\
x=\underbrace{(a+b) x}+\underbrace{(3 b-a)} \\
\quad a+b=1 \quad 3 b-a=0 \rightarrow a=\frac{3}{4} \quad b=\frac{1}{4}
\end{array}
\end{aligned}
$$

$$
\begin{aligned}
& \text { (1) } \int \frac{x+4}{x^{2}+6 x+10} d x=\int \frac{0.5(x+2)+1}{x^{2}+6 x+10} d x \\
& =\underbrace{\frac{1}{2} \int \frac{x+2}{x^{2}+6 x+10} d x}_{I_{1}}+\underbrace{\int \frac{d x}{x^{2}+6 x+10}}_{I_{2}} \\
& I_{1}=\frac{\ln \left|x^{2}+6 x+10\right|}{2}+c \\
& I_{2}=\int \frac{d x}{(x+3)^{2}+1}=\arctan (x+3)+c
\end{aligned}
$$

$$
\begin{aligned}
& \text { :'clp } \\
& \left(x^{2}+6 x+10\right)^{\prime}=\{2 x+6\} \\
& x+4=a(2 x+6)+b=2 a x+6 a+b \\
& a=0.5 \quad b=1 \longleftarrow 2 a=1 \quad 6 a+b=4 \\
& \int \frac{x+4}{x^{2}+6 x+10} d x=\frac{\ln \left|x^{2}+6 x+10\right|}{2}+\arctan (x+3)+c \quad \text { : p1o } 0 \delta
\end{aligned}
$$

(2)

$$
\begin{aligned}
& \int \frac{x}{2 x^{2}+x+3} d x=\int \frac{\frac{1}{4}(4 x+1)-\frac{3}{4}}{2 x^{2}+x+3} \\
& \left(2 x^{2}+x+3\right)^{\prime}=4 x+1 \\
& x=a(4 x+1)+b=4 a x+a+b \\
& =\underbrace{\frac{1}{4} \int \frac{4 x+1}{2 x^{2}+x+3} d x}_{I_{1}}-\underbrace{\frac{1}{4} \int \frac{d x}{2 x^{2}+x+3}}_{I_{2}} \\
& I_{1}=\frac{\ln \left|2 x^{2}+x+3\right|}{4}+C \\
& I_{2}=\frac{-1}{4} \int \frac{d x}{2\left(x^{2}+\frac{x}{2}+\frac{3}{2}\right)} \\
& x^{2}+\frac{x}{2}+\frac{3}{2}=\left(x+\frac{1}{4}\right)^{2}-\frac{1}{16}+\frac{3}{2} \\
& =\frac{-1}{4} \cdot \frac{1}{2} \int \frac{d x}{\left(x+\frac{1}{4}\right)^{2}+\frac{23}{16}}=\frac{-1}{8} \cdot \frac{1}{\sqrt{\frac{23}{16}}} \arctan \left(\frac{x+\frac{1}{4}}{\sqrt{\frac{23}{16}}}\right)+c \\
& =\frac{-\sqrt{16} \arctan \left(\frac{(x+0.25) \sqrt{16}}{\sqrt{23}}\right)}{8 \cdot \sqrt{23}}+C=\frac{-\arctan \left(\frac{4 x+1}{\sqrt{23}}\right)}{2 \sqrt{23}}+C \\
& \int \frac{x}{2 x^{2}+x+3} d x=\frac{\ln \left|2 x^{2}+x+3\right|}{4}-\frac{\arctan \left(\frac{4 x+1}{\sqrt{23}}\right)}{2 \sqrt{23}}+C \quad: \text { olvod }
\end{aligned}
$$

$$
\left.\begin{array}{r}
\frac{x^{2}-2 x+4}{-2]} \\
\frac{x^{3}+0+0-1}{-2 x^{2}+0} \\
\frac{-2 x^{2}-4 x}{4 x-1} \\
-4 x+8 \\
-9
\end{array}\right\}\left(\frac{x^{3}-1}{x+2} d x \quad\right. \text { sk aen }
$$

\includegraphics[max width=\textwidth, center]{2025_04_04_0bb41e73ad949df7df10g-3}\\
\includegraphics[max width=\textwidth, center]{2025_04_04_0bb41e73ad949df7df10g-3(1)}\\
\includegraphics[max width=\textwidth, center]{2025_04_04_0bb41e73ad949df7df10g-3(2)}\\
\includegraphics[max width=\textwidth, center]{2025_04_04_0bb41e73ad949df7df10g-3(3)}\\
\includegraphics[max width=\textwidth, center]{2025_04_04_0bb41e73ad949df7df10g-3(4)}

$$
\begin{aligned}
& \sup (T) \leq n C e \leq \inf (E) \\
& \inf (E)=\sup (T), \quad{ }_{a}^{b} f(x) d_{x}=\inf (E)=\sup (T) \text { лk フ(PC) }
\end{aligned}
$$

$$
\int_{a}^{b} f(x) \alpha_{x}=F(b)-F(a) \quad: k \lambda ग \text { nis'e } \longleftarrow
$$


\end{document}

\documentclass[10pt]{article}
\usepackage[utf8]{inputenc}
\usepackage[T1]{fontenc}
\usepackage{graphicx}
\usepackage[export]{adjustbox}
\graphicspath{ {./images/} }
\usepackage{amsmath}
\usepackage{amsfonts}
\usepackage{amssymb}
\usepackage[version=4]{mhchem}
\usepackage{stmaryrd}

\begin{document}
DION Suchlk*\\
\includegraphics[max width=\textwidth, center]{2025_04_04_79ca77055beaefdef237g-1(1)}

$$
\forall x \in[a, b] \quad F^{\prime}=f
$$

\begin{center}
\includegraphics[max width=\textwidth]{2025_04_04_79ca77055beaefdef237g-1}
\end{center}

$$
\begin{aligned}
& \int_{a}^{b} f(x) d x=F(b)-F(a)=\left.F(x)\right|_{a} ^{b} \quad \text { : ible } f \int_{e} \text { गNiPp }
\end{aligned}
$$

$$
\begin{aligned}
& \begin{array}{l}
\left.F_{a}^{x}=f(x) \quad, 7(t) d t\right)^{\prime}=(F(x)-F(a))^{\prime}=F^{\prime}(x)=f(x)
\end{array}
\end{aligned}
$$

(1) $\int_{1}^{2} x^{3} d x=\left.\frac{x^{4}}{4}\right|_{1} ^{2}=\frac{2^{4}}{4}-\frac{1^{4}}{4}=\frac{16-1}{4}=\frac{15}{4}$\\
(2) $\int_{0}^{1} \frac{x}{e^{x}} d x=\int_{0}^{1} x e^{-x} d x \quad\left|\begin{array}{ll}u^{\prime}=e^{-x} & v=x \\ u=-e^{-x} & v^{\prime}=1\end{array}\right| \rightarrow-\left.x e^{-x}\right|_{0} ^{1}+\int_{0}^{1} e^{-x} d x$

$$
=\left(-1 \cdot e^{-1}-\left((-0) \cdot e^{-0}\right)\right)-\left.e^{-x}\right|_{0} ^{1}=\frac{-1}{e}-\left(e^{-1}-e^{-0}\right)=\frac{-2}{e}+1
$$

$$
\begin{aligned}
& \int_{a}^{b} f(x) d x=-\int_{b}^{a} f(x) d x \\
& (F(a)-F(b))=-\int_{b}^{a} f(x) d x
\end{aligned}
$$

(1) $\int_{-1}^{1} \sqrt{2 x+2} d x=\int_{-1}^{1}(2 x+2)^{0.5} d x=\left.\frac{(2 x+2)^{1.5}}{1.5 \cdot 2}\right|_{-1} ^{1}=\frac{(2+2)^{1.5}-(-2+2)^{1.5}}{3}=\frac{8}{3}$\\
(2) $\int_{0}^{\pi}(\cos (3 x)-\sin (x)) d x=\frac{\sin (3 x)}{3}+\left.\cos (x)\right|_{0} ^{\pi}=\frac{0}{3}+-1-(0+1)=-2$\\
once ale'n\\
Sxdrlen de niat Con Coen

$$
\int_{-1}^{1}|x| d x=\int_{-1}^{0}(-x) d x+\int_{0}^{1} x d x=\left.\frac{-x^{2}}{2}\right|_{-1} ^{0}+\left.\frac{x^{2}}{2}\right|_{0} ^{1}=\left(\frac{-0}{2}-\frac{-1}{2}\right)+\left(\frac{1}{2}-\frac{0}{2}\right)=1 \quad: \quad \text { 'dp }
$$

\includegraphics[max width=\textwidth, center]{2025_04_04_79ca77055beaefdef237g-2(1)}\\
\includegraphics[max width=\textwidth, center]{2025_04_04_79ca77055beaefdef237g-2(2)}\\
\includegraphics[max width=\textwidth, center]{2025_04_04_79ca77055beaefdef237g-2}

$$
\begin{aligned}
& =\int_{0}^{a} f(t) d(t)+\int_{0}^{a} f(x) d(x)=2 \int_{0}^{a} f(x) d x
\end{aligned}
$$

$$
\begin{aligned}
& \int_{-\pi}^{\pi} \sin (x) d x
\end{aligned}
$$

$$
\begin{aligned}
& \text { - } \int_{a}^{b} c f(x) d x=c \int_{a}^{b} f(x) d x \quad \cdot \int_{a}^{b}\left(f(x)+g(x) d x=\int_{a}^{b} f(x) d x+\int_{a}^{b} g(x) d x\right. \\
& a<c<b e \text { ann } k \delta \int_{a} \int_{a}^{b} f(a)=\int_{a}^{\int} f\left(a x d x+\int_{a}^{b} f(x) d x \quad: \operatorname{sor} C .\right.
\end{aligned}
$$

\begin{center}
\includegraphics[max width=\textwidth]{2025_04_04_79ca77055beaefdef237g-3}
\end{center}

$$
\int_{a}^{b} g(x) d x \leq \int_{a}^{b} f(x) d x: 5 k
$$

$\left|\int_{a}^{b} f(x) d x\right| \leq \int_{a}^{b}|f(x)| d x$ ask $[a, b]$ rCã ri.Sianc (J.k $f$ ot

$$
0=\int_{a}^{b} 0 d x \leq \int_{a}^{b} f(x) d x \quad . \forall x \in[a, b] \quad 0 \leq f(x)!
$$

\begin{center}
\includegraphics[max width=\textwidth]{2025_04_04_79ca77055beaefdef237g-3(1)}
\end{center}

$$
\int_{a}^{b}(f(x)-g(x)) d x
$$

$$
\int_{a}^{c}[f(x)-g(x)] d x+\int_{c}^{b}[g(x)-f(x)] d x
$$

\includegraphics[max width=\textwidth, center]{2025_04_04_79ca77055beaefdef237g-3(4)}\\
\includegraphics[max width=\textwidth, center]{2025_04_04_79ca77055beaefdef237g-3(3)}\\
\includegraphics[max width=\textwidth, center]{2025_04_04_79ca77055beaefdef237g-3(2)}

$$
g(x)={\underset{4}{x}-2 x-2}_{x^{2}-2}^{n} \quad 23 p 1201
$$


\end{document}

\documentclass[10pt]{article}
\usepackage[utf8]{inputenc}
\usepackage[T1]{fontenc}
\usepackage{amsmath}
\usepackage{amsfonts}
\usepackage{amssymb}
\usepackage[version=4]{mhchem}
\usepackage{stmaryrd}
\usepackage{graphicx}
\usepackage[export]{adjustbox}
\graphicspath{ {./images/} }

\begin{document}
$$
\begin{aligned}
& {\left[\begin{array}{ll}
u^{\prime}=\sin (x) & v=e^{x} \\
u=-\cos (x) & v^{\prime}=e^{x}
\end{array}\right] \quad \int e^{x} \sin (x) d x} \\
& \longrightarrow-e^{x} \cos (x)+\int e^{x} \cos (x) d x\left[\begin{array}{ll}
u^{\prime}=\cos (x) & v=e^{x} \\
u=\sin (x) & v^{\prime}=e^{x}
\end{array}\right] \\
& \longrightarrow \int e^{x} \sin (x) d x=-e^{x} \cos (x)+e^{x} \sin (x)-\int e^{x} \sin (x) d x \\
& 2 \int e^{x} \sin (x) d x=e^{x}(\sin (x)-\cos (x)) \quad \int e^{x} \sin (x) d x=\frac{e^{x}(\sin (x)-\cos (x))}{2}+c \\
& {[f(g(x))]^{\prime}=f^{\prime}(g(x)) \cdot g^{\prime}(x)} \\
& \int \frac{g(f(x)) f^{\prime}(x)}{p \sec (J \cdot c} d x=G(\rho(x))+c \\
& \text { Pond. } G \text { asnos } g \text { fe onip pie nus } \\
& {[G(f(x))]^{\prime}=G^{\prime}(f(x)) \cdot \varphi^{\prime}(x)=g(f(x)) \cdot f^{\prime}(x)}
\end{aligned}
$$

(1) $\int \sin ^{2}(x) \cos (x) d x$

$$
\left[g(x)-x^{2} \quad f(x)=\sin (x)\right]
$$

$] \longrightarrow g(f(x))=\sin ^{2}(x)$ $f^{\prime}(x)=\cos (x)$

$$
\int \sin ^{2}(x) \cos (x) d x=\frac{\sin ^{3}(x)}{3}+c
$$

\begin{center}
\includegraphics[max width=\textwidth]{2025_04_04_812097e24b697489fdbdg-1}
\end{center}

$$
\begin{aligned}
& \int g(\rho(x)) \rho^{\prime}(x) d x=G(\rho(a))+c \quad \begin{array}{l}
t=\rho(x) \\
\int g(t) d t=G(t)+c
\end{array} \frac{d t}{d x}=\rho^{\prime}(x) \longrightarrow d t=\rho^{\prime}(x) d x
\end{aligned}
$$

(1) $\int \frac{\left(e^{x}-d x\right)}{e^{2 x}+1}\left\{\begin{array}{ll}t=e^{x} & \\ \frac{d y}{d x}=e^{x} & d t=e^{x} d x\end{array}\right\}\left\{\begin{aligned} d t & =\arctan (t)+c \\ t^{2}+1 & =\arctan \left(e^{x}\right)+c\end{aligned}\right.$\\
(2) $\int \sin ^{2}(x) \cos (x) d x\left\{\begin{array}{l}t=\sin (x) \\ d t / d x=\cos (x) \quad d t=\cos (x) d x\end{array}\right\}$\\
\includegraphics[max width=\textwidth, center]{2025_04_04_812097e24b697489fdbdg-2}\\
\includegraphics[max width=\textwidth, center]{2025_04_04_812097e24b697489fdbdg-2(1)}\\
(5) $f(x) e^{x^{2}}(d x)\left\{\begin{array}{l}t=x^{2} \\ d t / \alpha x=2 x\end{array} d t=2 x d x\right\} \quad \int \frac{e^{t} d t}{2}=\frac{1}{2} \int e^{t} d t=\frac{e^{t}}{2 t=c}+\frac{e^{2}}{2}+c$

$$
\begin{aligned}
& \frac{1}{\alpha} \int \alpha f(\alpha x+\beta) \alpha x=\frac{1}{\alpha} F(\alpha x+\beta)+c, \alpha \neq 0 \\
& \left\{\begin{array}{l}
t=\alpha x+\beta \\
\alpha t / \alpha x=\alpha \longrightarrow \alpha t=\alpha \alpha x
\end{array}\right\} \\
& \frac{1}{\alpha} \int f(t) \alpha t=\frac{1}{\alpha} F(t)+c=\frac{1}{\mid t=\alpha x+\beta}=\frac{1}{\alpha} F(\alpha x+\beta)+c \\
& \int \frac{f^{\prime}(x)}{f(x)} d x \quad\left\{\begin{array}{l}
t=f(x) \\
\frac{d y}{d x}=f^{\prime}(x) \quad d t=f^{\prime}(x) d x
\end{array}\right\} \\
& \int \frac{\alpha t}{t}=\ln |t|_{t=f(x)}+c=\ln |f(x)|+c
\end{aligned}
$$

(1) $\int \frac{e^{x}}{e^{x}+5} d x=\ln \left|e^{x}+5\right|+c$\\
(2) $\frac{1}{5} \int \frac{5 x^{4}}{x^{5}-9} d x=\frac{1}{5} \ln \left|x^{5}-9\right|+c$

$$
\text { (3) } \int \tan (x) d x=-\int \frac{-\sin (x)}{\cos (x)}=-\ln |\cos (x)|+c
$$

$$
\begin{aligned}
& \text { งอา เลวาก } \int \delta_{0} \\
& \int \frac{d x}{\sqrt{x}(1+\sqrt[3]{x})}\left\{\begin{array}{l}
x=t^{6} \\
\frac{d x}{d t}=6 t^{5}
\end{array} \quad d x=6 t^{5} d t\right\} \quad \int \frac{6 t^{5} d t}{\sqrt{t^{6}}\left(1+\sqrt[3]{t^{6}}\right)} \\
& \int \frac{6 t^{5} d t}{t^{6}\left(1+t^{2}\right)}=6 \int \frac{t^{2}}{1+t^{2}} d t=6 \int \frac{t^{2}+1-1}{1+t^{2}} d t=6 \int\left(1-\frac{1}{1+t^{2}}\right) d t \\
& =6(t-\arctan (t)+c)=6(\sqrt[6]{x}-\arctan (\sqrt[6]{x})+c) \\
& \int \sqrt{1-x^{2}} d x \quad\left\{\begin{array}{ll}
x & x=\sin (t) \\
t=\arcsin (t) \\
\frac{d x}{d t}=\cos (t) & d x=\cos (t) d t
\end{array}\right\} \quad \int \sqrt{1-\sin ^{2}(x)} \cos (t) d t \\
& \int \sqrt{\substack{\cos ^{2}(t)}} \cdot \cos (t) d t=\int \cos ^{2}(t) d t=\int \frac{\cos (2 t)+1}{2} d t=\frac{1}{2}\left[\int \cos (2 t) d t+\int d t\right] \\
& =\frac{1}{2}\left(\frac{\sin (2 t)}{2}+t+c\right)=\frac{1}{2}\left(\frac{2 \sin (t) \cos (t)}{t}+t+c\right) \\
& =\frac{1}{2}\left(x \cdot \sqrt{1-x^{2}}+\arcsin (x)+c\right)=\frac{x \sqrt{1-x^{2}}+\operatorname{arcsinn}(x)}{2}+c
\end{aligned}
$$


\end{document}

\documentclass[10pt]{article}
\usepackage[ngerman]{babel}
\usepackage[utf8]{inputenc}
\usepackage[T1]{fontenc}
\usepackage{amsmath}
\usepackage{amsfonts}
\usepackage{amssymb}
\usepackage[version=4]{mhchem}
\usepackage{stmaryrd}
\usepackage{graphicx}
\usepackage[export]{adjustbox}
\graphicspath{ {./images/} }

\begin{document}
$x^{3}+4 x^{2}+2 x+6$

$$
\frac{x^{4}+3 x^{3}-2 x^{2}+4 x+5}{x-1}
$$

: KNCIP

$$
x^{4}+3 x^{3}-2 x^{2}+4 x+5 \quad x-1
$$

$\frac{x^{4}-x^{3}}{-4 x^{3}-2 x^{2}+4 x+5}$

$$
\begin{aligned}
& \begin{array}{l}
4 x^{3}-2 x^{2}+4 x+5 \\
4 x^{3}-4 x^{2} \\
-2 x^{2}+4 x+5
\end{array}
\end{aligned}
$$

$$
\frac{2 x^{2}-2 x}{6 x+5}
$$

$$
\begin{aligned}
& \frac{-2 x}{-6 x+5} \\
& \left.\frac{6 x-6}{11}\right\} \text { ?nvic TD TN }
\end{aligned}
$$

! eine os ind\\
$\longrightarrow \lim _{x \rightarrow 1} \frac{x^{3}+2 x^{2}-10 x+7}{x-1}=" \frac{0}{0}$

$$
\text { so } \rightarrow \frac{\frac{x^{2}+3 x-7}{x^{3}+2 x^{2}-10 x+7 x-1}}{} \begin{aligned}
& \frac{x^{3}-x^{2}}{3 x^{2}-10 x+7} \\
& \\
& 3 x^{2}-3 x
\end{aligned}
$$

$=\lim _{x \rightarrow 1} \frac{(x-T)\left(x^{2}+3 x-7\right)}{(x-1)}=1+3-7=-3$

$$
\begin{array}{r}
3 x^{2}-3 x \\
-7 x+7 \\
-7 x+7 \\
0
\end{array}
$$

$\left.\frac{\frac{x^{3}-2 x}{x^{5}-x^{3}+x-1} x^{2}+1}{\frac{x^{5}+x^{3}}{-2 x^{3}+x-1}} \begin{array}{l}\frac{-2 x^{3}-2 x}{3 x-1}\end{array}\right\}$

$$
\begin{aligned}
x^{5}-x^{3}+x-1 & =\left(x^{2}+1\right)\left(x^{3}-2 x\right)+3 x-1 \\
& =\left(x^{2}+1\right)\left(x^{3}-2 x+\frac{3 x-1}{x^{2}+1}\right)
\end{aligned}
$$

\begin{center}
\includegraphics[max width=\textwidth]{2025_04_04_889feed8fdc1844b0a97g-2}
\end{center}

$$
\begin{aligned}
& B=\sin (x), A=\sin ^{-1}(x+h): \operatorname{jos} \quad \lim _{h \rightarrow 0} \frac{\arcsin (x+h)-\arcsin (x)}{h}: \operatorname{Sen} \delta \\
& \sin (B)=x \cdot \sin (A)=x+h \\
& \quad \sin (A)-\sin (B)=h
\end{aligned}
$$

$$
A-B \rightarrow 0 \text { yk } A \rightarrow B \text { sk } \sin (A)-\sin (B) \rightarrow 0: \text { inid reak }
$$

$$
\begin{gathered}
\operatorname{LapJ:~}_{\lim _{A \rightarrow B} \frac{A-B}{\sin (A)-\sin (B)}=\lim _{A \rightarrow B} \frac{\sin (x)-\sin (y)=2 \sin \left(\frac{x-y}{2}\right) \cos \left(\frac{x+y}{2}\right)}{2 \sin \left(\frac{A-B}{2}\right) \cos \left(\frac{A+B}{2}\right)}=\lim _{A \rightarrow B}\left[\frac{A-B}{2}\right.}^{\left.\left.\frac{\sin \left(\frac{A-B}{2}\right)}{2}\right) \cdot \frac{1}{\cos \left(\frac{A+B}{2}\right)}\right]} \\
=\lim _{A \rightarrow B} \frac{1}{\cos \left(\frac{A+B}{2}\right)}=\frac{1}{\cos (B) \sin } \\
=\frac{\cos ^{2}(x)+\sin ^{2}(x)=1 \quad 3 \cos (x)=\sqrt{1-\sin ^{2}(x)}}{\sqrt{1-\sin ^{2}(B)}}=\frac{1}{\sqrt{1-x^{2}}}
\end{gathered}
$$

(2)

$$
\begin{aligned}
& \lim _{h \rightarrow 0} \frac{\tan ^{-1}(x+h)-\tan ^{-1}(x)^{2 x}}{h} \\
&=\left.\lim _{h \rightarrow 0} \frac{\tan ^{-1}\left(\frac{x+h-x}{1+x(x+h)}\right)}{h}=\lim _{h \rightarrow 0} \frac{\tan ^{-1}(x)-\tan ^{-1}(y)=\tan ^{-1}\left(\frac{x-y}{1+x y}\right): \Omega \operatorname{lan}}{1+x(x+h)}\right) \\
&= \lim _{h \rightarrow 0}\left[\tan ^{-1}\left(\frac{h}{1+x(x+h)}\right) \div\left(h \cdot \frac{1}{1+x(x+h)} \cdot[1+x(x+h)]\right)\right]=\lim _{h \rightarrow 0} \frac{\tan ^{-1}(x)}{x}=1: \pi n y \\
& 1+x^{2}+x h
\end{aligned}
$$

(1) $\left(x^{2} \tan ^{-1}(x)\right)^{\prime}=2 x \tan ^{-1}(x)+x^{2} \cdot \frac{1}{1+x^{2}}=2 x \tan ^{-1}(x)+\frac{x^{2}}{1+x^{2}}$\\
(2) $\left(\cos ^{-1}\left(e^{-x^{2}}\right)\right)^{\prime}=\frac{-1}{\sqrt{1-\left(e^{-x^{2}}\right)^{2}}} \cdot\left(e^{-x^{2}}\right)^{\prime}=\frac{-1}{\sqrt{1-\left(e^{-x^{2}}\right)^{2}}} \cdot-2 x \cdot e^{-x^{2}}=\frac{2 x}{e^{x^{2}} \sqrt{1-e^{-2 x^{2}}}}$

$$
\text { (3) } \begin{aligned}
&\left.\left(\sin ^{-1}\left(\frac{x}{\sqrt{1+x^{2}}}\right)\right)^{\prime}=\frac{1}{1-\left(\frac{x}{\sqrt{1+x^{2}}}\right)^{2}} \cdot \frac{\left(1+x^{2}\right)^{\frac{1}{2}}-x \cdot 0 \cdot 5\left(1+x^{2}\right)^{\frac{-1}{2}} \cdot 2 x}{1+x^{2}}\right) * \frac{\left(1+x^{2}\right)^{\frac{1}{2}}}{\left(1+x^{2}\right)^{\frac{1}{2}}} \\
&= \frac{1}{1-\frac{x^{2}}{1+x^{2}}} \cdot \frac{1+x^{2}-x^{2}}{\left(1+x^{2}\right)^{\frac{3}{2}}}=\frac{1}{\sqrt{\frac{1+x^{2}-x^{2}}{1+x^{2}}}} \cdot \frac{1}{\left(1+x^{2}\right)^{\frac{3}{2}}}=\left(1+x^{2}\right)^{\frac{1}{2}} \cdot \frac{1}{\left(1+x^{2}\right)^{\frac{3}{2}}} \\
&=\frac{1}{1+x^{2}}
\end{aligned}
$$

(1)

$$
\begin{aligned}
\lim _{x \rightarrow 1^{-}} \frac{\sin ^{-1}(x)-\pi / 2}{\sqrt{1-x}}=\frac{" 0^{\prime \prime}}{0}{ }^{\prime} \delta 1 \delta & =\lim _{x \rightarrow 1^{-}} \frac{\frac{1}{\sqrt{1-x^{2}}}}{-1 \cdot 0.5(1-x)^{-0.5}}
\end{aligned}=\lim _{x \rightarrow 1^{-}} \frac{-2 \sqrt{1-x}}{\sqrt{1-x^{2}}} .
$$

(2)

$$
\begin{array}{r}
\lim _{x \rightarrow 0} \frac{\tan ^{-1}(x)-x}{x^{3}}=" \frac{0}{0} 0 \delta=\lim _{x \rightarrow 0} \frac{\frac{1}{1+x^{2}}-1}{3 x^{2}}=\lim _{x \rightarrow 0} \frac{1-1-x^{2}}{3 x^{2}\left(1+x^{2}\right)} \\
=\lim _{x \rightarrow 0} \frac{-x^{4}}{3 x^{2}\left(1+x^{2}\right)}=\frac{-1}{3}
\end{array}
$$

(3) $\lim _{x \rightarrow 0} \frac{\sin ^{-1}(x)+\cos ^{-1}(x)-\frac{\pi}{2}}{x^{2}}=\frac{0^{\prime \prime}}{0}=\lim _{x \rightarrow 0} \frac{\frac{1}{\sqrt{1-x^{2}}}+\frac{-1}{\sqrt{1-x^{2}}}}{2 x}=\lim _{x \rightarrow 0} \frac{0}{2 x}=0$


\end{document}

% This LaTeX document needs to be compiled with XeLaTeX.
\documentclass[10pt]{article}
\usepackage[utf8]{inputenc}
\usepackage{ucharclasses}
\usepackage{graphicx}
\usepackage[export]{adjustbox}
\graphicspath{ {./images/} }
\usepackage{amsmath}
\usepackage{amsfonts}
\usepackage{amssymb}
\usepackage[version=4]{mhchem}
\usepackage{stmaryrd}
\usepackage{polyglossia}
\usepackage{fontspec}
\setmainlanguage{hindi}
\setotherlanguages{english, arabic}
\IfFontExistsTF{Noto Serif Devanagari}
{\newfontfamily\hindifont{Noto Serif Devanagari}}
{\IfFontExistsTF{Kohinoor Devanagari}
  {\newfontfamily\hindifont{Kohinoor Devanagari}}
  {\IfFontExistsTF{Devanagari MT}
    {\newfontfamily\hindifont{Devanagari MT}}
    {\IfFontExistsTF{Lohit Devanagari}
      {\newfontfamily\hindifont{Lohit Devanagari}}
      {\IfFontExistsTF{FreeSerif}
        {\newfontfamily\hindifont{FreeSerif}}
        {\newfontfamily\hindifont{Arial Unicode MS}}
}}}}
\IfFontExistsTF{Noto Naskh Arabic}
{\newfontfamily\arabicfont{Noto Naskh Arabic}}
{\IfFontExistsTF{Al Bayan}
  {\newfontfamily\arabicfont{Al Bayan}}
  {\IfFontExistsTF{FreeSerif}
    {\newfontfamily\arabicfont{FreeSerif}}
    {\IfFontExistsTF{DejaVu Sans}
      {\newfontfamily\arabicfont{DejaVu Sans}}
      {\IfFontExistsTF{Tahoma}
        {\newfontfamily\arabicfont{Tahoma}}
        {\newfontfamily\arabicfont{Arial Unicode MS}}
}}}}
\IfFontExistsTF{CMU Serif}
{\newfontfamily\lgcfont{CMU Serif}}
{\IfFontExistsTF{DejaVu Sans}
  {\newfontfamily\lgcfont{DejaVu Sans}}
  {\newfontfamily\lgcfont{Georgia}}
}
\setDefaultTransitions{\lgcfont}{}
\setTransitionsForDevanagari{\hindifont}{\rmfamily}
\setTransitionsFor{Arabic}{\arabicfont}{\lgcfont}

\begin{document}
\includegraphics[max width=\textwidth]{2025_04_04_9b394a6c498e9f44540cg-1} . $f \pi k \quad \mid k 3 N . f(0)=0$

$$
-1-0+c=0
$$

$$
\begin{array}{r}
\left.\int f^{\prime}(x) d x=\int(-\cos (x)-\sin (x)+1) d x=-\sin (x)+\cos (x)+x+\alpha=f(x)\right\} f(0)=0 \text { o s|ip } \\
f(0)=\cos (0)-\sin (0)+0+c=0 \quad 1-0+0+\alpha=0 \quad \text { be } \\
1-1=-1
\end{array}
$$

$$
\begin{aligned}
& \left.\int f^{\prime \prime}(x)=\int(\sin (x)-\cos (x)) d x=-\cos (x)-\sin (x)+c=f^{\prime}(x)\right\} \quad f^{\prime}(0)=0 \quad \circ \quad \text { of } p^{\prime} \\
& f^{\prime}(0)=-\cos (0)-\sin (0)+c=0 \\
& \text { ob }
\end{aligned}
$$

$$
\begin{aligned}
& F(x)=\int f(x) d x
\end{aligned}
$$

$$
\begin{aligned}
& \left(e^{x}\right)^{\prime}=e^{x} \longrightarrow \int e^{x} d x=e^{x}+c \\
& (\ln (x))^{\prime}=\frac{1}{x} \longrightarrow \int \frac{d x}{x}=\ln |x|+c \\
& \longrightarrow \int \frac{(x-2)^{2}}{x} d x=\int \frac{x^{2}-4 x+4}{x} d x=\int\left(x-4+\frac{4}{x}\right) d x=\frac{x^{2}}{2}-4 x+4 \ln |x|+c \\
& \rightarrow \int e^{-3 x} d x=\frac{e^{-3 x}}{-3}+c=c-\frac{1}{3 e^{3 x}} \\
& \longrightarrow \int \frac{3^{2 x}+5^{x+1}}{4^{x}} d x=\int \frac{9^{x}+5 \cdot 5^{x}}{4^{x}}=\int\left[\left(\frac{9}{4}\right)^{x} d x+5 \int\left(\frac{5}{4}\right)^{x}\right] d x=\frac{\left(\frac{9}{4}\right)^{x}}{\ln (9 / 4)}+\frac{5\left(\frac{5}{4}\right)^{x}}{\ln (5 / 4)}+c \\
& \longrightarrow \int \frac{d x}{x^{2}+4 x+5}=\int \frac{d x}{(x+2)^{2}+1}=\arctan (x+2)+c \\
& \rightarrow \int \frac{d x}{x^{2}+4 x+6}=\int \frac{d x}{(x+2)^{2}+2}=\frac{\arctan \left(\frac{x+2}{\sqrt{2}}\right)}{\sqrt{2}}+C \\
& \longrightarrow \int \frac{d x}{\sqrt{x}-\sqrt{x+1}(\sqrt{x}+\sqrt{x+1})}=\int \frac{\sqrt{x+1}+\sqrt{x+1}}{x-x+1} d x=\int \sqrt{x} d x+\int \sqrt{x+1} d x=\frac{x^{\frac{3}{2}}}{3 / 2}+\frac{(x+1)^{\frac{3}{2}}}{3 / 2}+c \\
& =\frac{2\left(x^{\frac{3}{2}}+(x+1)^{3 / 2}\right)}{3}+c
\end{aligned}
$$

$$
\begin{array}{ll}
\int f^{\prime \prime}(x) d x=f^{\prime}(x) & f 2 d x=2 x+c \\
\int f^{\prime}(x) d x=f(x) & f(2 x-5) d x=x^{2}-5 x+\alpha
\end{array}
$$

$$
f^{\prime}(1)=2(1)+c=-3
$$

$$
f^{\prime}(x)=2 x-5
$$

\includegraphics[max width=\textwidth, center]{2025_04_04_9b394a6c498e9f44540cg-2}\\
when $x=1 \longrightarrow \quad y=-3(1)+5=2 \quad(1,2)$ (p)\\
so $\quad f(1)=1^{2}-5(1)+\alpha=2$

$$
f(x)=x^{2}-5 x+6
$$

$$
\begin{gathered}
\int \frac{x+2}{\left(9 x^{2}+42 x+49\right)^{\frac{1}{3}}} d x=\int \frac{x+2}{(3 x+7)^{\frac{2}{3}}} \quad \begin{array}{l}
x+2=a(3 x+7)+b=3 a x+7 a+b \quad: \quad \begin{array}{l}
1=3 a \quad \\
=\int \frac{1}{3}(3 x+7)-\frac{1}{3} \\
(3 x+7)^{\frac{2}{3}}
\end{array}=\frac{1}{3} \int \frac{3 x+7}{\left(3 x+77^{\frac{2}{3}}\right.} d x-\frac{1}{3} \int \frac{d x}{(3 x+7)^{\frac{2}{3}}} \\
=\frac{1}{3}\left[\int(3 x+7)^{\frac{1}{3}} d x-\int(3 x+7)^{\frac{-2}{3}} d x\right]=\frac{1}{3}\left[(3 x+7)^{\frac{4}{3}} \cdot \frac{1}{4 / 3 \cdot 8}-(3 x+7)^{\frac{1}{3}} \cdot \frac{1}{\frac{1}{3} \cdot 7 \cdot 7}\right]+c
\end{array} \\
=\frac{(3 x+7)^{\frac{4}{3}}}{12}-\frac{(3 x+7)^{\frac{1}{3}}}{3}+c
\end{gathered}
$$

$(f(x) \cdot g(x))^{\prime}=f^{\prime}(x) g(x)+f(x) g^{\prime}(x)$ integation $f(f(x) \cdot g(x))^{\prime} d x=\int f^{\prime}(x) g(x) d x+\int f(x) g^{\prime}(x) d x$

$$
\int f^{\prime}(x) g(x) d x=f(x) g(x)-\int f(x) g^{\prime}(x) d x
$$

(1)

$$
\left.\int \frac{x}{e^{x}} d x=\int x e^{-x} d x\left|\begin{array}{ll}
u^{\prime}=e^{-x} & v=x \\
u=-e^{-x} & v^{\prime}=1
\end{array}\right| \longrightarrow-x e^{-x}+\int e^{-x} d x\right]
$$

(2) $\int x(3 x+1)^{20} d x$

$$
\begin{aligned}
\left.\begin{array}{ll}
u^{\prime}=(3 x+1)^{28} & v=x \\
u=\frac{(3 x+1)^{21}}{63} & v^{\prime}=1
\end{array} \right\rvert\, & \rightarrow \frac{x(3 x+1)^{21}}{63}-\frac{1}{63} \int(3 x+1)^{21} d x \\
& =\frac{x(3 x+1)^{21}}{63}-\frac{(3 x+1)^{22}}{63 \cdot 66}+c
\end{aligned}
$$

(3) $\int e^{x} \sin (x) d x$ $\left.\begin{array}{ll}u^{\prime}=e^{x} & v=\sin (x) \\ u=e^{x} & v^{\prime}=\cos (x)\end{array} \right\rvert\, \rightarrow e^{x} \sin (x)-\int e^{x} \cos (x) d x$

$$
\begin{aligned}
& \begin{array}{ll}
\left.\begin{array}{ll}
u^{\prime}=e^{x} & v=\cos (x) \\
u=e^{x} & v^{\prime}=-\sin (x)
\end{array} \right\rvert\, \longrightarrow e^{x} \sin (x)-\left(e^{x} \cos (x)+\int e^{x} \sin (x) d x\right) \\
\int e^{x} \sin (x) d x=e^{x} \sin (x)-e^{x} \cos (x)-\int e^{x} \sin (x) d x
\end{array} \\
& 2 \int e^{x} \sin (x) d x=e^{x}(\sin (x)-\cos (x)) \quad \int e^{x} \sin (x) d x=\frac{e^{x}(\sin (x)-\cos (x))}{2}
\end{aligned}
$$

$$
\text { : } 1 \text { กล3กา תle }
$$

(1) $\int \frac{d x}{x \ln ^{6}(x)}\left|t=\ln (x) \quad d t=\frac{d x}{x}\right| \quad \int \frac{d t}{t^{6}}=\frac{-1}{5 t^{5}}+t=\ln (x)$\\
(2) $\int \frac{e^{x}}{\sqrt{1-e^{2 x}}} d x \quad\left|t=e^{x} \quad d t=e^{x} d x\right| \int \frac{d t}{\sqrt{1-t^{2}}}=\arcsin (t)+c$\\
$=\arcsin$ $=\arcsin \left(e^{x}\right)+c$\\
(3) $\int \tan (2 x) d x=\int \frac{\sin (2 x)}{\cos (2 x)} d x \quad|t=\cos (2 x) \quad d t=(2) \sin (2 x) d x|$

$$
=-\frac{1}{2} \int \frac{d t}{t}=\frac{-\ln |t|}{2}+c=\frac{-\ln |\cos (2 x)|}{2}+c
$$

(4) $\left.\int e^{\sqrt{x}} d x \quad t=\sqrt{x} \quad d t=\frac{d x}{2 \sqrt{x}}\right\} \left.\begin{aligned} d x & =2 \sqrt{x} d t \\ & =2 t d t\end{aligned} \right\rvert\, \int e^{t} 2 t d t=2 \int e^{t} t d t$

$$
\left|\begin{array}{ll}
u^{\prime}=e^{t} & v=t \\
u=e^{t} & v^{\prime}=1
\end{array}\right| \longrightarrow 2\left(t e^{t}-\int e^{t} d t\right)=2\left(t e^{t}-e^{t}\right)+c=2 e^{\sqrt{x}}(\sqrt{x}-1)+c
$$

(5) $\left.\int e^{\sqrt{5 x-7}} d x \left\lvert\, t=\sqrt{5 x-7} \quad d t=\frac{5 d x}{2 \sqrt{5 x-7}}\right.\right\} d x=\frac{2 t d t}{5} \left\lvert\, \frac{2}{5} \int e^{t} t d t\right.$

$$
=\frac{2}{5}\left(t e^{t}-e^{t}\right)+c=\frac{2 e^{\sqrt{5 x-7}}}{5}(\sqrt{5 x-7}-1)+c
$$

$$
\begin{gathered}
\text { (6) } \int \frac{\ln ^{2}(x)}{x^{2}} d x \quad \left\lvert\, \begin{array}{cc}
t=\ln (x) & d t=\frac{d x}{t} \\
x=e^{t} & \int \frac{t^{2}}{e^{t}} d t\left|\begin{array}{ll}
u^{\prime}=e^{-t} & v=t^{2} \\
u=-e^{-t} & v^{\prime}=2 t
\end{array}\right| \\
\longrightarrow-t^{2} e^{-t}+2 \int t e^{-t} \alpha t \quad\left|\begin{array}{ll}
u^{\prime}=e^{-t} & v=t \\
u=-e^{-t} & v^{\prime}=1
\end{array}\right| \\
\rightarrow-t^{2} e^{-t}+2\left(-t e^{-t}+\int e^{-t} \alpha t\right)=-t^{2} e^{-t}-2 t e^{-t}-2 e^{-t}+c \mid t=\ln (x) \\
=e^{-\ln (x)}\left(-\ln ^{2}(x)-2 \ln (x)-2\right)+c=\frac{-1}{x}\left(\ln ^{2}(x)+2 \ln (x)+2\right)+c
\end{array} .\right.
\end{gathered}
$$

$h(x)+\cos ^{-1}(h(x))=x \quad$ |лر $o k \quad \lim _{x \rightarrow 1} h^{\prime}(x) \quad \pi k$\\
$)=x \quad: h(x)=1$ ok $\quad h^{\prime}(x)-\frac{h^{\prime}(x)}{\sqrt{1-h^{2}(x)}}=1$ $x=1$\\
$1 \longleftarrow x$ nek $1 \longleftarrow h(x)$ ble

$$
\begin{gathered}
h^{\prime}(x)\left(1-\frac{1}{\sqrt{1-h^{2}(x)}}\right)=1 \quad h^{\prime}(x)=\frac{1}{1-\frac{1}{\sqrt{1-h^{2}(x)}}} \\
\lim _{x \rightarrow 1} \frac{1}{1-\frac{1}{\sqrt{1-\frac{h^{\prime}(x)}{\longrightarrow}}}}=0 \\
\xrightarrow{1}+\infty
\end{gathered}
$$


\end{document}