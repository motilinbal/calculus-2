So, $\frac{x^3-1}{x+2} = x^2 - 2x + 4 - \frac{9}{x+2}$.
Now integrate term by term:
\begin{align*} I &= \int \left( x^2 - 2x + 4 - \frac{9}{x+2} \right) \dd x \\ &= \int x^2 \dd x - \int 2x \dd x + \int 4 \dd x - \int \frac{9}{x+2} \dd x \\ &= \frac{x^3}{3} - 2\frac{x^2}{2} + 4x - 9 \ln|x+2| + C \\ &= \frac{x^3}{3} - x^2 + 4x - 9 \ln|x+2| + C \end{align*}
\end{example}

\section{The Definite Integral}

We now shift our focus from finding antiderivatives (indefinite integrals) to the concept of the definite integral, which has a fundamental connection to the geometric problem of finding the area under a curve.

\subsection{Motivation: The Area Problem}

Consider a non-negative continuous function $f(x)$ defined on a closed interval $[a, b]$. How can we determine the exact area $A$ of the region bounded by the curve $y=f(x)$, the x-axis, and the vertical lines $x=a$ and $x=b$?

The idea, dating back to antiquity but formalized significantly later, is to approximate this area using shapes whose areas are easy to calculate, namely rectangles.

\subsection{The Darboux Approach to Integration}

One rigorous way to define the area, and more generally the definite integral, is the Darboux approach (named after Gaston Darboux). It relies on approximating the area from below and above using rectangles.

\begin{definition}[Partition]
A \textbf{partition} $P$ of the interval $[a, b]$ is a finite set of points $\{x_0, x_1, x_2, \dots, x_n\}$ such that $a = x_0 < x_1 < x_2 < \dots < x_n = b$. This partition divides $[a, b]$ into $n$ subintervals $[x_{i-1}, x_i]$ for $i=1, \dots, n$. The length of the $i$-th subinterval is $\Delta x_i = x_i - x_{i-1}$.
\end{definition}

Now, assume $f$ is a bounded function on $[a, b]$ (continuity implies boundedness). For each subinterval $[x_{i-1}, x_i]$, let:
\begin{itemize}
    \item $m_i = \Inf \{ f(x) \mid x \in [x_{i-1}, x_i] \}$ (the infimum, or minimum value if $f$ is continuous, of $f$ on the subinterval).
    \item $M_i = \Sup \{ f(x) \mid x \in [x_{i-1}, x_i] \}$ (the supremum, or maximum value if $f$ is continuous, of $f$ on the subinterval).
\end{itemize}

\begin{definition}[Lower and Upper Darboux Sums]
Given a function $f$ bounded on $[a, b]$ and a partition $P=\{x_0, \dots, x_n\}$ of $[a, b]$:
\begin{itemize}
    \item The \textbf{Lower Darboux Sum} of $f$ with respect to $P$ is:
      \[ L(f, P) = \sum_{i=1}^n m_i \Delta x_i \]
      This represents the sum of areas of rectangles inscribed under the curve.
    \item The \textbf{Upper Darboux Sum} of $f$ with respect to $P$ is:
      \[ U(f, P) = \sum_{i=1}^n M_i \Delta x_i \]
      This represents the sum of areas of rectangles circumscribed about the curve.
\end{itemize}
For any partition $P$, it's clear that $L(f, P) \le A \le U(f, P)$, where $A$ is the true area (if $f \ge 0$).
\end{definition}

Intuitively, as we make the partition finer (i.e., make the subintervals smaller), the lower sums should increase (or stay the same) and the upper sums should decrease (or stay the same), both getting closer to the actual area.

\begin{definition}[Lower and Upper Darboux Integrals]
Let $f$ be bounded on $[a, b]$.
\begin{itemize}
    \item The \textbf{Lower Darboux Integral} of $f$ from $a$ to $b$ is the supremum of all lower sums, taken over all possible partitions $P$ of $[a, b]$:
      \[ \underline{\int_a^b} f(x) \dd x = \Sup \{ L(f, P) \mid P \text{ is a partition of } [a, b] \} \]
    \item The \textbf{Upper Darboux Integral} of $f$ from $a$ to $b$ is the infimum of all upper sums, taken over all possible partitions $P$ of $[a, b]$:
      \[ \overline{\int_a^b} f(x) \dd x = \Inf \{ U(f, P) \mid P \text{ is a partition of } [a, b] \} \]
\end{itemize}
It can be shown that for any bounded function $f$, $\underline{\int_a^b} f(x) \dd x \le \overline{\int_a^b} f(x) \dd x$.
\end{definition}

\begin{definition}[Darboux Integrability]
A bounded function $f$ is said to be \textbf{Darboux integrable} (or simply integrable) on $[a, b]$ if its lower and upper Darboux integrals are equal:
\[ \underline{\int_a^b} f(x) \dd x = \overline{\int_a^b} f(x) \dd x \]
If $f$ is integrable, this common value is called the \textbf{definite integral} (or Darboux integral) of $f$ from $a$ to $b$, denoted by:
\[ \int_a^b f(x) \dd x \]
\end{definition}

\begin{remark}[Notation Interpretation]
The notation $\int_a^b f(x) \dd x$ is very suggestive. The integral sign $\int$ is an elongated 'S', standing for 'Sum'. $f(x)$ represents the height of a rectangle at point $x$, and $dx$ represents an infinitesimally small width. The definite integral can be thought of as the sum of the areas of infinitely many infinitesimally thin rectangles from $x=a$ to $x=b$.
\end{remark}

\subsection{Conditions for Integrability}

Which functions are actually integrable? Fortunately, many common functions are.

\begin{theorem}[Integrability of Continuous Functions]
If a function $f$ is continuous on the closed interval $[a, b]$, then $f$ is Darboux integrable on $[a, b]$.
\end{theorem}

\begin{theorem}[Integrability of Piecewise Continuous Functions]
If a function $f$ is bounded on $[a, b]$ and is continuous except for at most a finite number of points in $[a, b]$, then $f$ is Darboux integrable on $[a, b]$.
\end{theorem}

These theorems cover most functions encountered in introductory calculus.

\begin{remark}[Non-Integrable Functions]
Not all functions are integrable. The classic example is the Dirichlet function, defined on $[0, 1]$ as $f(x) = 1$ if $x$ is rational and $f(x) = 0$ if $x$ is irrational. For any partition of $[0, 1]$, every subinterval contains both rational and irrational numbers. Thus, $m_i = 0$ and $M_i = 1$ for all $i$. This leads to $L(f, P) = 0$ and $U(f, P) = (1-0) = 1$ for all partitions $P$. Therefore, $\underline{\int_0^1} f(x) \dd x = 0$ but $\overline{\int_0^1} f(x) \dd x = 1$. Since the lower and upper integrals are not equal, the Dirichlet function is not Darboux (or Riemann) integrable.
\end{remark}

\section{The Fundamental Theorem of Calculus (FTC)}

The definition of the definite integral via Darboux sums (or the equivalent Riemann sums) is conceptually important but computationally impractical for most functions. The incredible breakthrough, discovered independently by Newton and Leibniz, provides the crucial link between differentiation (antiderivatives) and integration (area/definite integrals).

\begin{theorem}[The Fundamental Theorem of Calculus, Part 2]
Let $f$ be integrable on $[a, b]$. If $F$ is any antiderivative of $f$ on $[a, b]$ (i.e., $F'(x) = f(x)$ for all $x$ in $[a, b]$), then
\[ \int_a^b f(x) \dd x = F(b) - F(a) \]
\end{theorem}

\begin{remark}[Notation]
The expression $F(b) - F(a)$ is often denoted as $[F(x)]_a^b$ or $F(x) \big|_a^b$.
\end{remark}

\begin{remark}[Significance]
The FTC is truly fundamental because it connects the two main branches of calculus:
\begin{enumerate}
    \item Differential calculus (finding rates of change, slopes, derivatives).
    \item Integral calculus (finding areas, accumulation, definite integrals).
\end{enumerate}
It tells us that to find the definite integral (a concept defined via sums and limits related to area), we can instead find an antiderivative (a concept related to differentiation) and simply evaluate it at the endpoints. This makes calculating definite integrals vastly easier for most functions where we can find an antiderivative.
\end{remark}

\begin{remark}[FTC Part 1]
There is another part to the FTC, often called Part 1, which relates differentiation and integration in a different way: If $f$ is continuous on $[a, b]$, then the function $G(x) = \int_a^x f(t) \dd t$ is differentiable on $(a, b)$ and $G'(x) = f(x)$. This states that the process of integrating $f$ and then differentiating returns the original function $f$.
\end{remark}

We will use the FTC Part 2 extensively to evaluate definite integrals in upcoming examples and applications. The process will typically be:
1. Find an antiderivative $F(x)$ for the integrand $f(x)$.
2. Evaluate $F(b) - F(a)$.

\section{Administrative Notes}

\begin{itemize}
    \item \textbf{Tutorials:} Please attend the tutorial sessions. They are highly recommended for practice and clarification.
    \item \textbf{Class Schedule:} Please note that we do \textbf{not} have a class meeting on Tuesday. Our regular schedule continues on Wednesday and Thursday.
\end{itemize}

\end{document}